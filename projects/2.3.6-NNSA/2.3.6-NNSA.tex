\subsection{\stid{6} \nnsa}\label{subsect:nnsa}

\textbf{End State:} In the end, the goal is to have software used by the NNSA/ATDM national security
applications and associated exascale facilities be hardened for
production use and made available to the larger ECP community.

\subsubsection{Scope and Requirements}
The NNSA ST L3 area was created in FY20, although the projects included
have all been part of the ECP before its creation. The capabilities of these
software products remains aligned with the other ST
L3 areas from which they were derived, but are managed separately for
non-technical reasons out of scope of this document.

The resulting products in this L3 area 
are open source, important or critical to the success of the NNSA
National Security Applications, and are used (or potentially used) in
the broader ECP community. The products in this L3 span the scope of
the rest of ST (Programming Models and Runtimes, Development Tools,
Math Libraries, Data Analysis and Vis, and Software Ecosystem), and
will be coordinated with those other L3 technical 
areas through a combination of existing relationships and
cross-cutting efforts such as the ST SDKs and E4S.  

\subsubsection{Objectives}

The objective of these software products is to support the
development of new from-scratch applications within the NNSA that were
started just prior to the founding of the ECP under the ATDM 
program element within NNSA and
ASC. While earlier incarnations of these products may have been more
research-focused, by the time of the ECP ST restructuring in 2019 that
resulted in this L3 area, these products are in regular use by their
ATDM applications, and have matured to the point where they are ready
for use within the broader open source community.

\subsubsection{Plan}

NNSA ST products are developed with and alongside a broader
portfolio of ASC products in an integrated program, and are planned
out at high level in the annual ASC Implementation Plan, and in detail
using approved processes within the home institution/laboratory. They
are scoped to have 
resources sufficient for the success of the NNSA mission, as well as a
modicum of community support (e.g., maintaining on GitHub, or answering
occasional questions from the community).

For ECP products not part of the NNSA portfolio that have critical
dependencies on these products, there are often other projects within
ECP that provide additional funding and scope for those activities. In
those cases, there may be additional information within this document
on these products.


\subsubsection{Risks and Mitigation Strategies}

%A primary risk within this L3 area is that the 2020 ASC Level 1 milestone,
%designed as a capstone for the ATDM initiative and a decision point
%for the ultimate transition of those applications into the core ASC
%portfolio, will fail due to the inadequacy of these software
%products. While not all of them are on the critical path to
%application success (instead focusing on productivity enhancements for
%end users, or analysis functionality), it is expected that first and
%foremost they will contribute to the success of that milestone, as any
%subsequent ASC milestones and decision points about the ultimate fate
%of those applications. Mitigation is to use other ASC funding to bolster these
%efforts as needed.

One risk associated with the NNSA ST projects is the programming environment
of the El Capitan system. The programming environment on this system
will be a departure from what the NNSA software teams have used before,
so there is a risk that it will present challenges that cause delays
in porting the software to the system. That said, the probability of this risk is low because 
the programming environment of El Capitan will also be installed on DOE predecessor 
machines so it is likely that it challenges will be identified and addressed by
the time of El Capitan. NNSA ST projects can mitigate this risk by 
evaluating their software on the predecessor systems to identify challenges early.

Another risk associated with the NNSA ST L3 is that the projects rely
on multiple sources of funding outside of ECP. The budget priorities of 
those external sources may not always be aligned with those of ECP. In general,
this risk is low because the L4 leads strive to align their project goals across 
all funding sources. However, it is possible that funding expected to be leveraged
to develop a feature to later be used for ECP purposes may be dropped. If this 
occurs, the L4 leads will need to mitigate the situation according to their 
individual project needs, perhaps by renegotiating deliverable time lines.

Another risk is that others in the community will pick up these
products as open source, and expect additional support beyond the
scope of the primary NNSA mission. If those dependent products are within the
ECP, the main mitigation is to use ASCR contingency funding to provide
additional development and support - potentially through support of teams
outside of the home institution. If those dependent products are in
the broader community, then mitigations are generally outside of the scope
of the ECP, although each NNSA lab typically has some sort of project
(or possibly even a policy) on how to deal with external demands on
open source products.



