\subsubsection{\stid{3.15} Sake Sub-project: Kokkos Kernels} 
\paragraph{Overview} 
The Kokkos Kernels~\footnote{https://github.com/kokkos/kokkos-kernels} subproject primarily focuses on performance portable kernels
for sparse/dense linear algebra, graphs, and machine learning, with emphasis on
kernels that are key to the performance of
several ECP applications. We work closely with ECP applications to identify the
performance-critical kernels and develop portable algorithms for these kernels.
The primary focus of this subproject is to support ECP application needs, develop new
kernels needed with an emphasis towards software releases, tutorials, boot camps
and user support. The Kokkos Kernels project also works closely with several vendors
(AMD, ARM, Cray, Intel, and NVIDIA) as part of both the ECP collaborations and
NNSA's Center for Excellence efforts. These collaborations will  enable vendor solutions
that are targeted towards ECP application needs.

\paragraph{Key  Challenges}
There are several challenges in allowing ECP applications move to the hardware architectures
announced in the next few years. We highlight the four primary challenges here:
\begin{enumerate}
\item 
The next three supercomputers that will be deployed will have
three different accelerators from AMD, Intel and NVIDIA. While we have been expecting diversity of architectures, three
different architectures in such a short timeframe adds pressure on the portability
solutions such as Kokkos Kernels to optimize and support the kernels on all the platforms.
\item
The design of several ECP applications and a software stack that rely on a component-based
approach results in an extremely high number of kernel launches on the accelerators, which
results in the latency costs becoming the primary bottleneck in scaling the applications.
\item
The change in the needs of applications from device-level kernels to smaller team-level kernels. Vendor
libraries are not ready for such a drastic change in software design.
\item 
The reliance of ECP applications on certain kernels that do not port well to the  accelerator architectures.
\end{enumerate}

These challenges require a collaborative effort to explore new algorithmic choices,
working with the vendors to incorporate ECP needs into their library plans, to develop
portable kernels from scratch, and to deploy them in a robust software ecosystem. The Kokkos
Kernels project will pursue all of these choices in an effort to address these challenges.

\paragraph{Solution Strategy}

Our primary solution strategy to address these challenges are:
\begin{enumerate}
    \item \textbf{Codesign portable kernels with vendors and applications:}
    We rely on codesign of Kokkos Kernels implementations for
    specializations that are key to the performance of ECP applications. This
    requires tuning kernels even up to the problem sizes that are of interest
    to our users. Once we have developed a version, we provide these
    to all the vendors so their teams can optimize these kernels even
    further in vendor-provided math libraries.
   \item \textbf{Emphasis on software support and usability:}
	The Kokkos Kernels project devotes a considerable amount of time working with
	ECP applications, integrating the kernels into application codes, tuning
	for application needs, and providing tutorials and user support. We invest
	in delivering a robust software ecosystem that serves the
	needs of diverse ECP applications on all platforms of interest.
   \item \textbf{Invest in algorithmic research to reduce latency costs and new accelerator focused approaches:} 
   To resolve latency cost issues, the Kokkos Kernels team is considering several solutions from computer science
   perspectives and also from algorithmic applied mathematics perspectives. 
   For example, from a computer science perspective, we are focusing on the
   use of streams or other latency reducing techniques such as cuda graphs. From
   the applied mathematics perspective we are developing new algorithms such
   as cluster-based approaches for preconditioners, such as Gauss-Seidel precondtioners,
   to reduce the number of kernel launches. 
\end{enumerate}

\paragraph{Recent Progress}
\begin{enumerate}
\item Kokkos Kernels has implemented and released its HIP backend to support deployment on early
access platform Spock. While all of the kernels are available for this backend further work on
performance improvement and vendor TPL support are forth coming.
\item New graph algorithm for distance 2 maximum independent set and coarsening have been developed.
They provide support for the coarsening set of multigrid algorithms on GPUs, and of the applications
dependent on it such as ExaWind and EMPIRE.
\item Recently the SYCL backend has been enabled in Kokkos Kernels, it is not currently supporting all
algorithms (sparse tests are disabled) but provides initial capabilities and will be released in upcoming
Kokkos Kernels version 3.5.0.
\item Kokkos Kernels team has developed and integrated tutorial materials to the Kokkos tutorials. The
tutorials are maintained as a common resource for the entire Kokkos ecosystem.
\end{enumerate}

\paragraph{Next Steps}

Kokkos Kernels team is focused on:
\begin{enumerate}
	\item \textbf{A major software release}: Kokkos ecosystem 4.0 release 
	will be available to the ECP applications in FY 2022. This includes several new 
	kernels that are requested by ECP applications, performance improvements of
	kernels that are already being used by ECP applications and further support for
        new SYCL and OpenMP Target to enable applications to be tested on Intel architectures.
	\item \textbf{Developing new sparse batched kernels}: Kokkos Kernels team is working on
        sparse batched kernels that allow application to solve small linear system that might
        exist at integration point embeded in larger systems of equations. Such algorithms will
        support ECP applications Pele and XGC.
	\item \textbf{Collaboration with vendors}: Kokkos Kernels team is working with vendor
	libraries team to incorporate ECP application needs in the vendor library roadmap.
	Conversation regarding batched algorithms is on going and collaboration are being
        actively leveraged to further prepare the library for ExaScale platforms readiness.
\end{enumerate}
