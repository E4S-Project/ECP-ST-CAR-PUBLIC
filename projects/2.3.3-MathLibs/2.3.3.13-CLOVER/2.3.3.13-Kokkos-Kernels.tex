\subsubsection{\stid{3.13} CLOVER Sub-project Kokkos Kernels} 
\paragraph{Overview} 
The Kokkos Kernels~\footnote{https://github.com/kokkos/kokkos-kernels} subproject primarily focuses on performance portable kernels
for sparse/dense linear algebra, graphs, and machine learning, with emphasis on
kernels that are key to the performance of
several ECP applications. We work closely with ECP applications to identify the
performance-critical kernels and develop portable algorithms for these kernels.
The primary focus of this subproject is to support ECP application needs, develop new
kernels needed with an emphasis towards software releases, tutorials, boot camps
and user support. The Kokkos Kernels project also works closely with several vendors
(AMD, ARM, Cray, Intel, and NVIDIA) as part of both the ECP collaborations and
NNSA's Center for Excellence efforts. These collaborations will  enable vendor solutions
that are targeted towards ECP application needs.

\paragraph{Key  Challenges}
There are several challenges in allowing ECP applications move to the hardware architectures
announced in the next few years. We highlight the four primary challenges here:
\begin{enumerate}
\item 
The next three supercomputers that will be deployed will have
three different accelerators from AMD, Intel and NVIDIA. While we have been expecting diversity of architectures, three
different architectures in such a short timeframe adds pressure on the portability
solutions such as Kokkos Kernels to optimize and support the kernels on all the platforms.
\item
The design of several ECP applications and a software stack that rely on a component-based
approach results in an extremely high number of kernel launches on the accelerators, which
results in the latency costs becoming the primary bottleneck in scaling the applications.
\item
The change in the needs of applications from device-level kernels to smaller team-level kernels. Vendor
libraries are not ready for such a drastic change in software design.
\item 
The reliance of ECP applications on certain kernels that do not port well to the  accelerator architectures.
\end{enumerate}

These challenges require a collaborative effort to explore new algorithmic choices,
working with the vendors to incorporate ECP needs into their library plans, to develop
portable kernels from scratch, and to deploy them in a robust software ecosystem. The Kokkos
Kernels project will pursue all of these choices in an effort to address these challenges.

\paragraph{Solution Strategy}

Our primary solution strategy to address these challenges are:
\begin{enumerate}
    \item \textbf{Codesign portable kernels with vendors and applications:}
    We rely on codesign of Kokkos Kernels implementations for
    specializations that are key to the performance of ECP applications. This
    requires tuning kernels even up to the problem sizes that are of interest
    to our users. Once we have developed a version, we provide these
    to all the vendors so their teams can optimize these kernels even
    further in vendor-provided math libraries.
   \item \textbf{Emphasis on software support and usability:}
	The Kokkos Kernels project devotes a considerable amount of time working with
	ECP applications, integrating the kernels into application codes, tuning
	for application needs, and providing tutorials and user support. We invest
	in delivering a robust software ecosystem that serves the
	needs of diverse ECP applications on all platforms of interest.
   \item \textbf{Invest in algorithmic research to reduce latency costs and new accelerator focused approaches:} 
   To resolve latency cost issues, the Kokkos Kernels team is considering several solutions from computer science
   perspectives and also from algorithmic applied mathematics perspectives. 
   For example, from a computer science perspective, we are focusing on the
   use of streams or other latency reducing techniques such as cuda graphs. From
   the applied mathematics perspective we are developing new algorithms such
   as cluster-based approaches for preconditioners, such as Gauss-Seidel precondtioners,
   to reduce the number of kernel launches. 
\end{enumerate}

\paragraph{Recent Progress}
\begin{enumerate}
\item Kokkos Kernels has developed team-level linear algebra kernels for several BLAS and LAPACK
operations so that ECP applications can use these foundational operations within their device level code.
This key design championed by the Kokkos Kernels team is becoming more common place with several
vendors adopting such a design. This design reduces synchronization overhead and encourages reuse
of the data in the memory hierarchy between several BLAS or LAPACK operations with team level
synchronization. This has resulted in better performance in applications like SPARC CFD simulation.  
\item Kokkos Kernels preconditioners such as Symmetric Gauss Seidel precondtioners are integrated
into the Exawind application. The multicoloring based Symmetric Gauss Seidel precondtioner has resulted in
up to 4.5x improvement in overall solve time of the SGS solver. 
\item Kokkos Kernels BLAS and sparse linear algebra kernels were integrated into the spectral partitioner
of the Exagraph project. Using Kokkos Kernels results in faster spectral partitioning than vendor provided
implementations. This was the result of careful tuning of the kernels for Exagraph needs.
\item Kokkos Kernels team has developed and integrated tutorial materials to the Kokkos tutorials. The
tutorials are maintained as a common resource for the entire Kokkos ecosystem.
\end{enumerate}

\paragraph{Next Steps}

Kokkos Kernels team is focused on:
\begin{enumerate}
	\item \textbf{A major software release}: Kokkos ecosystem 3.0 release 
	will be available to the ECP applications in FY 2020. This includes several new 
	kernels that are requested by ECP applications, performance improvements of
	kernels that are already being used by ECP applications, support for new architectures
	such as ARM based systems, and several software changes such as support of 
	standalone CMake.  
	\item \textbf{Developing new kernels to reduce synchronization costs}: Kokkos Kernels
	team is working on kernels that reduce the number of kernels launches by focusing on
	block-based approaches. This will allow further performance in ECP applications
	such as Exawind and EMPIRE.
	\item \textbf{Collaboration with vendors}: Kokkos Kernels team is working with vendor
	libraries team to incorporate ECP application needs in the vendor library roadmap.
	Several changes from vendors are expected in FY20. These changes will be added to
	Kokkos Kernels so ECP applications can get access to the improvements. 
\end{enumerate}
