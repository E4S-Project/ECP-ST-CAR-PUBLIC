\subsubsection{\stid{6.01} LANL ATDM Data and Visualization} 

\paragraph{Overview} 
The LANL ATDM Data and Visualization project develops scalable
systems software for the generation, analysis, and management of data
produced by ECP applications. This project is essential for ECP because
of the volume of data exascale applications will generate.

\paragraph{Key Challenges}
Interfacing to a large number of ECP application with the Cinema software and the management of the volumous data from these applications.

\paragraph{Solution Strategy}
%\textit{Describe your basic strategy for addressing the challenges.}
The LANL ATDM Data and Visualization ECP project is focused on delivering new
systems software capabilities for creating, analyzing, and managing data for
Exascale scientific applications and Exascale data centers. We have identified
4 distinct areas that are in need of specific improvements.

MarFS is the only campaign storage system within the DOE complex and is
also the only HPC storage system built using scale-out principles with affordable
SMR hard drives. The LANL monitoring stack is leveraging available open source
software to build dashboards for monitoring both the data center and
scientific application performance. Finally, the application level software
technologies, Cinema~\cite{cinema:Ahrens:SC14} and HXHIM, are being developed in coordination with
LANL's ECP application NGC to ensure that data collected during the simulation
execution is of appropriate frequency, resolution, and viewport for later
analysis and visualization by scientists. Cinema is an innovative way of
capturing, storing and exploring extreme scale scientific data. Cinema is
essential for ECP because it embodies approaches to maximize insight from
extreme-scale simulation results while minimizing data footprint 

\paragraph{Recent Progress}
%\textit{Describe what you have done recently.  It would be good to have some
%kind of figure or diagram in this section.}
The MarFS file system is currently deployed as the campaign storage tier
with over 60PB of capacity currently under management in our secure computing
environment. Recent progress includes the development of a new highly
resilient backend based on nested parity. We have also extended our top-level
erasure approach to use RDMA operations for more efficient coding and data movement.

The LANL monitoring stack has been successfully deployed into multiple
computing enclaves within LANL's HPC facility, including LANL's secure
computing environment. Our approach currently supports data ingest and
analysis by system administrators, data analysts, and code teams with multiple
dashboards for each role. We continue to refine our security approach to
ensure that new monitoring monitoring dashboards comply with the requirements
of LANL's CCB, the voting body for ensuring that all LANL deployments are
secured appropriately. The use of our monitoring system by application teams
(which monitor information that includes classified data) has required a
highly granular approach to storage and access roles -- but also makes it more
broadly useful and provides direct benefit to the code teams and users.

\begin{figure}[htb]
	\centering
	\includegraphics[width=6in]{projects/2.3.6-NNSA/2.3.6.01-LANL-ATDM/hxhim-main}
	\caption{\label{fig:hxhim} Relevant components of the HXHIM
	embeddable service. The client library is provided as a set of API
	calls while the server capability is provided by a thread running
	within the application. Communication uses the Margo and Mercury RPC
	layers to provide efficient support for remote key-value access.}
\end{figure}

Recent progress on HXHIM, a key-value store for HPC platforms, includes the
integration of a new transport layer based on Margo and Mercury (projects
under development by the ECP data libs project). The fundamental architecture
of HXHIM now leverage new support for using a high-performance RPC package
(Mercury) layered beneath a C++ wrapper for Margo (called Thallium). HXHIM
provides bulk (multi-key) primitives that enable efficient use of HPC
interconnects and typical scientific storage workloads.


Recent Cinema work has focused on development of analysis capability, Exascale workflows and working with ECP STs such as ALPINE to enable in situ and post processing production of Cinema databases.  Currently Cinema is available in situ via ParaView Catalyst and Ascent and available post hoc via ParaView and VisIt.   New capabilities provide scientists more options in analyzing and exploring the results of large simulations by providing a workflow that 1) detects features in situ, 2) captures data artifacts in Cinema databases, 3) promotes post-hoc analysis of the data, and 4) provides data viewers that allow interactive, structured exploration of the resulting artifacts. 
%
In our most recent milestone, we ran two end-to-end simulation pipelines with ECP applications at scale to generate Cinema databases and ran Cinema-based workflows with Cinema algorithms to produce secondary set of artifacts.  We ran (1) Nyx integrated with Ascent, running the ALPINE adaptive sampling algorithm; and (2) SW4 integrated with Ascent, running a VTK-m isocontour algorithm.  

\paragraph{Next Steps}
%\textit{Describe what you are working on next.}
Cinema is focusing on outreach to ECP applications to identify new application workflows that can be reasonably made efficient and working on new analysis methods for Cinema users.

\begin{figure}[htb]
	\centering
	\includegraphics[width=4in]{projects/2.3.6-NNSA/2.3.6.01-LANL-ATDM/cinema-sw4-example.png}
	\caption{
		Screen capture of a browser-based viewer displaying the results of a analysis workflow using an SW4 isocontour Cinema database.  
	\label{fig:cinema-sw4example}
	}
\end{figure}

HXHIM, MarFS, and the monitoring infrastructure have all been descoped
from ECP though the work will continue as part of LANL's Computational Systems
and Software Environments effort. We do not expect this change to alter the
trajectory of any of the descoped projects.
