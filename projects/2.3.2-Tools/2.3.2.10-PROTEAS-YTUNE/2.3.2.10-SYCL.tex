\subsubsection{\stid{2.10} PROTEAS-TUNE - SYCL}\label{s:sycl}

\paragraph{Overview}
OpenCL is an open standard maintained by the Khronos group. It offers programming portability across a wide range of software and hardware for graphics processing units (GPUs), multi-core processors (CPUs), and other accelerators. As opposed to the OpenCL programming model in which host and device codes are written in two languages, the SYCL standard specifies a cross-platform abstraction layer that enables programming of heterogeneous computing system using standard C++. It can combine host and device codes for an application in a type-safe way to improve development productivity. SYCL is a promising programming model for exascale computing. The relevant topics are migration from CUDA only to SYCL, performance and expressiveness of SYCL programs, maturity of SYCL compilers, and accesses to SYCL programs.


The PROTEAS-TUNE objectives for SYCL are:
\begin{enumerate}
\item Develop a SYCL suite comprised of kernels from open-source benchmarks, scientific, and machine learning applications
\item Evaluate the performance of these kernels with contemporary SYCL implementations on heterogeneous computing platforms
\item Understand the impacts of SYCL compilers and computing platforms upon the performance gaps of these kernels
\item Propose SYCL features that can improve functional and performance portability 
\item Engage with vendors, facilities, universities, and communities for the development of SYCL applications and compilers
\end{enumerate}

 

\paragraph{Key Challenges}
Acknowledging CUDA's established presence in high-performance computing, SYCL has been striving for a portability-enhancing path for a wider set of platforms. While SYCL can achieve functional portability, it does not solve performance portability. To address the challenge, understanding the applications, programming models, SYCL features, SYCL compilers, and the architectures of heterogeneous computing platforms are critical. There are many scientific and AI applications. Major programming models for the target platforms are CUDA, HIP, OpenMP, and SYCL. Major SYCL features are extension to OpenCL C, single source, USM and buffer styles, and asynchronous programming. Major SYCL compilers are DPC++ with OpenCL and Level Zero backends, DPC++ with CUDA and HIP support, hipSYCL, and ComputeCpp. AMD, Intel, and Nvidia GPUs are different in their computing architectures. CPUs, GPUs, and FPGAs have fundamentally different architectures. The combination of applications, languages, features, toolchains, and architectures will characterize the performance of a SYCL program.


\paragraph{Solution Strategy}
\begin{enumerate}
\item Develop a diverse set of SYCL programs with compute- and memory-bound kernels for performance analysis  
\item Have a good understanding of the characteristics of these programs 
\item Evaluate the performance of SYCL programs with the latest SYCL compilers on computing platforms
\item Analyze the performance of SYCL kernels through performance profilers
\item Identify and summarize these performance differences
\item Engage with SYCL compiler developers to fix bugs and improve kernel performance
\item Collaborate with PROTEAS teams and SYCL developers on performance optimization and tuning
\end{enumerate}


\paragraph{Recent Progress}
\begin{enumerate}
\item Investigated the use of vendor and academic conversion tools by evaluating CUDA portability with HIPCL and DPCT \cite{dpct}.
\item Investigated the performance of integer sum reduction in SYCL and CUDA on GPUs \cite{reduction}.
\item Developing 200+ SYCL programs in the open-source GitHub repository (\url{https://github.com/zjin-lcf/HeCBench.git}).
\end{enumerate}


\paragraph{Next Steps}

\begin{enumerate}
\item Continue developing SYCL programs with codes of interest to ECP.
\item Evaluate kernel performance with SYCL compilers on target platforms.
\item Analyze the performance of SYCL kernels through performance profilers.
\item Sum up the optimization techniques for SYCL kernels on target platforms.
\item Engage with SYCL developers to fix bugs and improve performance and portability. 
\item Support compiler installation, feature requests, and bug reporting for ECP users.
\end{enumerate}

%eof
