\subsection{\stid{5} \ecosystem}\label{subsect:ecosystem}

\textbf{End State:} A production-ready software stack delivered to our facilities, vendor partners, and the open source HPC community.

\subsubsection{Scope and Requirements}
The focus of this effort is on the ``last mile'' delivery of software that is intended to be supported by DOE Facilities and/or vendor offerings. The scope of this effort breaks down into the following key areas:
\begin{itemize}
	\item Hardening and broad ST and facility adoption of Spack for easy build of software on all target platforms
        \item Delivery of formal software releases (E4S) in multiple packaging formats technologies -- from-source builds, modules, and containers
	\item Oversight of the ST SDKs developed in all five ST L3 areas, with a goal of ensuring the SDKs are deployed as production-quality products at the Facilities, and available to the broader open-source HPC community through coordinated releases
	\item Development of testing infrastructure (e.g., Continuous Integration) in collaboration with HI 2.4.4 (Software Deployment at the Facilities) for use by ECP teams at the Facilities
	\item Development and hardening of methods for software deployment through the use of container technology
	\item Development and hardening of a toolkit of reusable components for scientific workflow management systems
	\item Informal partnerships with the Linux Foundation's OpenHPC project for potential broader deployment of ST technologies in the OpenHPC ecosystem
\end{itemize}

A major goal of ST is to ensure that applications can trust that ST products will be available on DOE Exascale systems in a production-quality state, which implies robust testing, documentation, and a clear line of support for each product. This will largely be an integration effort building on both the SDKs project elements defined in each ST L3 area, and tight collaboration and coordination with the HI L3 area for Deployment of Software on Facilities (WBS 2.4.4). We will work to develop, prototype, and deliver foundational infrastructure for technologies such as continuous integration and containers in tight collaboration with our DOE facility partners. The ultimate goal is ensuring that the ECP software stack is robustly supported, as well as finding a reach into the broader HPC open-source community -- both of which provide the basis for long-term sustainability required by applications, software, Facilities, and vendors who rely upon these products.

Spack has broad adoption in the open-source community as an elegant solution toward solving many of the challenges presented by building software with many dependencies. Spack is one of the most visible outward-facing products in this L3 area, and is the basis for the SDK and E4S efforts.

\subsubsection{Assumptions and Feasibility}
Success in this effort will require a coordinated effort across the entire hardware and software stack –- in particular with HI 2.4.4 (Delivery of Software to Facilities) and in some cases, our vendor partners.  This cooperation is a critical first step in enabling our goals, and this area will drive toward ensuring those partnerships can flourish for mutual gain.

Given the project timelines and requirements of production systems at our Facilities, we do not envision a wholly new software stack as a feasible solution. We do however recognize that in many cases the software of today's HPC environments will very likely need to either be evolved or extended to meet the mission goals. This will require first, proof-of-concept on existing pre-Exascale hardware, and ultimately adoption of technologies by system vendors where required, and by other application and software teams.

\subsubsection{Objectives}
This area will focus on all aspects of integration of the ECP software stack embodied in E4S and the development of SDK Community Policies, and building the workflows community and deploying a toolkit of hardened, reusable components for workflow management systems, with a focus on putting the infrastructure in place (in partnership with HI and the SDKs) for production-quality software delivery through technologies such as Spack, continuous integration, and containers. 

\subsubsection{Plan}
Version 21.08 of the E4S was released in August 2021, which includes 88 ST products that have Spack packages. This release is downloadable from DockerHub under the ecpe4s area.  The E4S Spack Build Cache now includes binaries for ppc64le as well as x86\_64 and includes over 53,000 total binaries.  The E4S containers now support custom images for ECP applications, such as WDMApp, ExaWind, and Pantheon.  A regular cadence of E4S releases will continue, with updated ST products, broader facility adoption, and potentially inclusion in vendor offerings.

In close coordination with E4S, a number of SDKs are being developed across the other L3 ST areas, building on the years of experience the xSDK (Math Libraries).  These SDKs will become a prime vehicle for our delivery strategy, while also providing ST products with a standard set of community policies aimed at robust production-ready software delivery.  In October 2020, Version 1 of the E4S Community Policies was announced. The E4S Community Policies will serve as membership criteria for E4S member packages.  The E4S Community Policies will continue to evolve as we work toward release of Version 2 of them.

Spack continues to gain penetration across the ECP, and will be the de facto delivery method for ST products building from source. We provide Spack packaging assistance for ST users and DOE Facilities, and are developing new capabilities for Spack that enable automated deployments of software at Facilities, in containerized environments, and as part of continuous integration. A new effort to support running test suites within Spack environments via the {\tt spack test} functionality is underway and will be integrated with GitLab continuous integration to display dashboards for E4S tests.  Concurrently, we are developing technologies and best practices that enable containers to be used effectively at Facilities, and are pushing to accelerate container adoption within ECP.  

We filled a gap in the ST portfolio by instantiating a new project on scientific workflows, which is currently in an initial development phase.  Within this project, we are establishing the ExaWorks toolkit by assembling shared components from existing workflow projects. The ExaWorks toolkit will provide a robust, well-tested, documented, and scalable set of components that can be combined to enable diverse teams to produce scalable and portable workflows for a wide range of exascale applications. Importantly, the project will not create a new workflow system nor does it aim not to replace the many workflow solutions already deployed and used by scientists, but rather it will provide well engineered and scalable components which can be leveraged by new and existing workflows.

\subsubsection{Risks and Mitigation Strategies}
\begin{itemize}
	\item Deploying E4S on unknown architectures -- use Spack for deployment to decrease installation complexity
        \item Keeping updated versions of ST and dependent software in synch after initially achieving SDK interoperability
	\item Delays in deploying a common CI infrastructure lead to subsequent delays in an integrated software release
	\item Multiple container technologies in flight will make it hard to come to agreement on a ``common'' looking solution; may not be possible to generate containers that are both portable and performant
	\item ECP Application Reliance on workflow management systems that may not be scalable and performant at exascale -- mitigate by adopting robust, well-tested, and scalable components 
	\item OpenHPC partnership is ill-defined, and unfunded
	\item Sustainability of ECP ST capabilities after ECP has ended
\end{itemize}

\subsubsection{Future Trends}

SDKs will gain further traction in their communities as the benefits of 
interoperability and community policies are demonstrated.  We believe these processes will
become embedded into the communities and become one of the lasting legacies of ECP and
critical for sustainability beyond ECP.

Software deployments will continue to become more complex, especially when we require optimized 
builds for the unique and complicated exascale architectures.  Keeping dependencies updated and 
the software tested on these systems using continuous integration will tax the resources at 
the Facilities.  Software testing that includes interoperability and scalability tests will 
require further resources, both in terms of people to write the tests and the hours to
regularly run them.  These put greater emphasis on using and updating Spack as a
solution strategy for large collections of software and tight coordination with
HI and Facilities on CI infrastructure and resources.

Containers will become more popular and usable as a way to package the entire 
environment necessary to run an application on the exascale machines, thereby managing some of the 
complexity of an application deployment.  We expect that performance of an application within a 
container will be nearly as fast or faster than running the application on bare metal.  Application
build time will be reduced by using the associated build caches.

Workflows will continue to become more complex to complete their science missions, requiring
orchestration of many applications and scripts, executed at various scales across many 
different resource types, and often reliant on machine learning algorithms for guidance.
We expect that hardening workflow management systems and building a community centered
around robust and scalable components will be foundational for addressing these
complexities.  Moreover, we expect that container-based scientific workflows will 
begin to take off as we transition from demonstrations of applications at scale 
to performing science with them.

