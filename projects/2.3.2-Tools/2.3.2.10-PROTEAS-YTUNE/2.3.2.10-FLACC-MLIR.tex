\subsubsection{\stid{2.10} PROTEAS-TUNE - FLACC and MLIR: Creating and Maintaining OpenACC in LLVM/Flang}\label{s:flacc}

\paragraph{Overview}
Heterogeneous and manycore processors (e.g., multicore CPUs, GPUs,
Xeon Phi, etc.) are becoming the de facto architectures for current
HPC platforms and future Exascale platforms.  These architectures are
drastically diverse in functionality, performance, programmability,
and scalability, significantly increasing the complexity that ECP app
developers face as they attempt to fully utilize available hardware.

A key enabling technology pursued as part of PROTEAS is OpenACC.
While OpenMP has historically focused on shared-memory multi-core,
OpenACC was launched in 2010 as a portable programming model for
heterogeneous accelerators.  Championed by institutions like NVIDIA,
PGI, and ORNL, OpenACC has evolved into one of the most portable and
well recognized programming models for accelerators today.

Despite the importance of OpenACC, the only non-academic open-source OpenACC
compiler cited by the OpenACC website is GCC \cite{openaccOrgTools}.
However, GCC has lagged behind commercial compilers, such as PGI's, in
providing production-quality support for the latest OpenACC specifications
\cite{openACCValidationSuite}.  Moreover, GCC is known within the compiler
community to be challenging to extend and, especially within the DOE, is
losing favor to Clang and LLVM for new compiler research and development
efforts.

Recent efforts to build a Fortran counter-part to Clang in LLVM project have
been accelerated and big chunk of the Flang have been upstreamed.
Directive-based programming model are heavily used in Fortran applications ported
to accelerators. Unlike C and C++, Fortran doesn not have many alternatives.

FLACC proposes to develop a prototype OpenACC 3.0 implementation in Flang based
on MLIR to fill this gap. As the implementation of the OpenMP target offload
feature in Flang does not have a clear path, this work will help in this regard
by sharing code in the MLIR dialects and lowering sections with the Flang and
SOLLVE projects.

\paragraph{Key Challenges}

\begin{enumerate}

\item \textbf{OpenACC Support:} Developing production-quality,
      standards-conforming OpenACC compiler and runtime support is a large
      undertaking. Complicating that undertaking further is the need for
      optimization strategies that are competitive with existing commercial
      compilers, such as PGI's, which have been developed over many years
      since before the conception of the OpenACC standard.

\item \textbf{MLIR:} Flang and the OpenACC support for it rely on the MLIR
      project for the IR. MLIR has been upstreamed to
      the core LLVM project in early 2020 and it is still actively under
      development. Flang will be the first core frontend relying on MLIR.

\item \textbf{Runtime:} LLVM does not include an OpenACC runtime but only one
      for OpenMP at the moment. This runtime can be generalized to support
      missing OpenACC features. This generalization needs to be accepted by the
      current OpenMP community.

\item \textbf{OpenMP Stability:} As we plan to generalize the OpenMP runtime to
      support OpenACC, we will also rely on various part of the runtime that
      are already here. There has been some concern on the stability of the
      current OpenMP runtime implementation and especially the
      textit{libomptarget} responsible for the target offload part.

\end{enumerate}

\paragraph{Solution Strategy}

~
\vspace{-1em}

\begin{enumerate}

\item Flacc design follows a similar design as the OpenMP implementation for
      Flang. This design includes the following aspects:

\begin{enumerate}
\item An OpenACC MLIR dialect part of the core MLIR project.
\item A lowering from the Flang AST to a mix of FIR and OpenACC MLIR dialect.
\item A progressive lowering from MLIR to LLVM IR with runtime call.
\end{enumerate}

\end{enumerate}

\paragraph{Recent Progress}

\begin{enumerate}
\item OpenACC 3.0 parser is upstreamed to the Flang front-end. It covers the
      full specification and also implements the unparsing feature of Flang.
\item Semantic checking for OpenACC 3.0 is also upstreamed in Flang. While
      implementing this part, a new TableGen backed for directive-based language
      has been contributed upstreamed. This is used by OpenACC and OpenMP for
      both Clang and Flang.
\item Discussed Flacc at various ECP, HPC, and LLVM venues.
\end{enumerate}


\paragraph{Next Steps}

\begin{enumerate}
\item Continue the definition of the OpenACC MLIR dialect and complete the
      lowering to it.
\item Integration of Flacc runtime support with OpenMPIRBuilder being developed in the broader ECP Project.
\item Develop mapping of OpenACC runtime calls to OpenMP based on early work in Clacc.
\item Continue adding support to lower from MLIR to LLVM IR and runtime call.
\end{enumerate}
