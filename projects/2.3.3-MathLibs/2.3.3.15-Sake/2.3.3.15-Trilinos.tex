\subsubsection{\stid{3.15} Sake: Trilinos/PEEKS} \label{subsubsect:trilinos}
\paragraph{Overview} 
Trilinos is a large and widely used toolkit for scientific computing, with many users both at DOE labs, in academia, and in industry. 
This project is focused on making Trilinos ready for exascale. One part is to port a core set of Trilinos packages to relevant architectures 
(including NVIDIA, AMD, and Intel GPUs). The other part is to design algorithms that work well on accelerators and at large scale (the focus of 
the PEEKS sub-project).


\paragraph{Key  Challenges}
Developing preconditioned iterative solvers for the US flagship supercomputers 
deployed in ECP, we acknowledge three major challenges coming from the hardware 
architecture:
\begin{enumerate}
\item 
Fine-grained parallelism in a single node that has to be exploited efficiently 
by the iterative solver and the preconditioner.
\item
Rising communication and synchronization cost as the
computational power is growing much faster than memory power, resulting in 
increased pressure on the bandwidth of all cache/memory levels.
\end{enumerate}

All challenges require the redesign of existing iterative solvers with respect 
to higher parallelism, % within all building blocks
a reduced number of 
communication and synchronization points, favoring computations over 
communication, and adopting multiprecision algorithms for efficient hardware 
utilization. 

\paragraph{Solution Strategy}

The primary thrusts of the PEEKS project are:
\begin{enumerate}
   \item \textbf{Pipelined and CA Krylov methods:} 
    	We realize pipelined and 
	communication-avoiding Krylov methods in production-quality code, and 
	we are actively collaborating with the ECP ExaWind project to integrate 
        our new features into their application~\cite{Yamazaki-lowsynch}. 
	\item \textbf{Memory Precision Decoupling:}  In collaboration with the ECP 
	xSDK multiprecision effort, we are working on sparse linear algebra 
	iterative methods and preconditioners that reduce runtime by compressing 
	data before invoking memory operations, such as the adaptive precision 
	block-Jacobi preconditioner~\cite{toms_anzt} and the compressed basis 
	GMRES~\cite{aliaga2020compressed}. 
\end{enumerate}

\paragraph{Recent Progress}
\begin{enumerate}
\item  TODO
\end{enumerate}

\paragraph{Next Steps}

Our next efforts are:
\begin{enumerate}
	\item TODO
\end{enumerate}
