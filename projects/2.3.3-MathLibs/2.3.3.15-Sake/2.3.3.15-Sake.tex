\subsubsection{\stid{3.15} Sake} \label{subsubsect:sake}

\paragraph{Overview} 

Modern simulation codes running on high performance computing (HPC) machines often rely heavily on multiple libraries to provide core capabilities such as 
meshing, mathematical algorithms, I/O services and more. This approach is highly productive as it allows domain experts to focus on their core technical 
contributions. The Sake project focuses on the design and development of performance portable mathematical libraries within the Trilinos project and the 
Kokkos ecosystem. Specifically, the team provides new implementations of linear algebra methods optimized for the architectures planned for the upcoming exascale 
systems while using interfaces already defined in Trilinos allowing applications a smooth transition toward exascale readiness.

Sake has two software products: Trilinos and KokkosKernels. It is organized into three subprojects, Trilinos, KokkosKernels, and PEEKS, where 
PEEKS and KokkosKernels were formerly part of CLOVER, whereas the Trilinos subproject is new in ECP.
