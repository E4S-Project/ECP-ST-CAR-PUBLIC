\subsubsection{\stid{1.08} Legion}

\paragraph{Overview}
This project focuses on the development, hardening and support of the
Legion Programming System (\url{https://legion.stanford.edu}) with a
focus on Exascale platforms and workloads that can benefit from
Legion's features.  At a fundamental level our focus is on the key
capabilities (e.g. correctness, performance, scalability) of an
alternative programming model for ECP that seeks to expose additional
levels of parallelism in applications.  In addition, Legion also
enables a separation of concerns of the implementation of an
application from how it is mapped onto a given system architecture.

Our efforts are currently focused on addressing bugs, refactoring the
implementation for improved stability, performance and scaling,
extending support for the selected exascale platforms (Aurora and
Frontier), and also expanding the feature set as needed for both
application and platform nuances.

The Legion Programming System is freely available with an Apache-based
open source license and is hosted at GitLab:

\begin{quote} 
  \url{https://gitlab.com/StanfordLegion/legion}
\end{quote}

\paragraph{Key Challenges}
While Legion addresses a number of key challenges in improving system
utilization, some aspects of platform portability, and is becoming
more widely used, it is still a new programming system and therefore
there is a cost to rewriting applications.  This aspect makes
significant adoption a risk within ECP and additional effort is being
taken to improve stability and find unique use cases.  aspects of performance and scaling to
match aspects of more mature technologies.

We have focused much of our efforts on emerging use cases that are
related to machine learning and data-centric workloads.  These domains
are much easier to have a substantial impact as the application codes
rely on external tools (e.g. TensorFlow, Python, etc.) vs. years of
established code written in MPI and/or OpenMP.  We are already seeing
clear benefits of focusing our efforts in this direction. This has
helped us to increase our overall impact as well focus on areas of
adoption across more specialized application needs in support of
machine learning and other data-centric workloads.

\paragraph{Solution Strategy}
As a collaboration between LANL, Stanford University and R\&D
efforts at NVIDIA, SLAC, Facebook, MIT, and others.  We are providing
the overarching implementation of Legion that captures the most stable
(correct and feature complete) version of the programming system.  In
addition, we are actively looking for opportunities to educate the
community about Legion and other data-centric and task-based
approaches to programming.

We have continued working with ExaFEL (AD 2.2.4.05) and the CANDLE
project (AD 2.2.4.03) to provide support for Legion.  We also provide
support and software releases related to the efforts going on within
LANL's ATDM Programming Models and Runtimes project (part of ST
2.3.6.01), that refine, identify needs and requirements that are in
support of Ristra (LANL's National Security application AD 2.2.5.01).
Our project includes management of the current repository and
quarterly, or more frequent, releases of Legion to the broader
community.  We are also supporting approaches that support Legion
inter-operation with other languages and programming systems --
e.g. MPI, OpenMP, Kokkos, Fortran, Python.

We have continued our work on improving performance and scaling of
training deep learning applications.  In particular, we are working
closely with CANDLE's requirements for ML training and inference on
large DNNs. Our most recent progress is discussed below. 

\paragraph{Recent Progress}

We continue to discover and address both performance and scalability
issues in the runtime.  In addition, for use cases within ECP, and
also a growing set of users outside of ECP, we have continued to
identify and address bugs and other issues (e.g. missing features).

As mentioned above, our we continue to focus on improving training
times for CANDLE's DNN use cases and also improving developer productivity
when using Flexflow (the DNN training layer built on top of Legion).  
We are specifically working to make sure the feature set of FlexFlow system
provides the necessary functionality and as part of this we have recently
completed a Python layer for FlexFlow that provides support for the Keras
(TensorFlow) interface.  This allows TensorFlow programs to be run using
FlexFlow with very minimal changes to the original Python code.  FlexFlow
is now available as open source under an Apache License:

\begin{quote} 
  \url{https://github.com/flexflow/FlexFlow} 
\end{quote}

We are now starting work on new training networks with CANDLE and also
working on providing a PyTorch compatibility interface to provide the same
ability to run widely used Python applications for machine learning using
Flow and Legion as the underlying runtime systems. 

Finally, we currently have a start on supporting both AMD and Intel
GPUs.  After AMD deprecated the CUDA Driver API in their HIP software
layer we had to rewrite a portion of the underlying implementation to
re-establish AMD support.  This initial work is complete and Legion
applications are running successfully on currently available AMD GPUs.
Similar efforts are underway for Intel systems using the oneAPI
interface.  The initial implementation has run into a few issues
within Intel's software stack that we are waiting to be resolved.  We
are on target to successfully run on Intel's stack by late in 2020 or
early 2021.  We have also completed an additional lower-level
MPI-based transport layer underneath Legion to provide a level of risk
mitigation at the lower levels of the Legion runtime (Realm).

\paragraph{Next Steps}

Our plans for the next year are to continue focusing on the challenges
presented by the upcoming exascale system architectures and on
hardening and improving the overall performance and scalability of
Legion.  These efforts will specifically begin to focus on AMD and
Intel hardware with an eye towards performance at the node level
(limited by the availability and stability of appropriate hardware
resources).

We will continue to work on the Python interfaces for Legion with a
focus on the Keras and PyTorch features requested by CANDLE.  This
will include seeking out and improving our outreach.  Our regular open
source releases of Legion and FlexFlow will continue.  Further work will
need to be done to focus on bug fixes, improving capabilities, improving
developer productivity, and addressing performance issues on both existing
and upcoming platforms.



