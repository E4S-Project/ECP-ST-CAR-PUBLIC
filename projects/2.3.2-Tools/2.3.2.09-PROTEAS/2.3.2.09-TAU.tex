\subsubsection{PROTEAS | TAU Performance System}\label{subsubsect:tau}

\paragraph{Overview} 
The TAU Performance System is a versatile profiling and tracing toolkit that supports performance instrumentation, measurement, and analysis.  Figure~\ref{figure:tau} gives an example of using TAU's parallel profile analysis tool, ParaProf. It is a robust, portable, and scalable performance tool for use in parallel programs and systems over several technology generations. It is a ubiquitous performance tool suite for shared-memory and message-passing parallel applications written in C++, C, Fortran, Java, Python, UPC, and Chapel. In the PROTEAS project, TAU is being extended to support compiler-based instrumentation for the LLVM C, C++, and Fortran compilers using higher-level intermediate language representation. TAU is also targeting support for performance evaluation of directive based compilation solutions using OpenARC and it will support comprehensive performance evaluation of NVM based HPC systems.  Through these and other efforts, our objective to better support parallel runtime systems such as OpenMP, OpenACC, Kokkos, and CUDA in TAU. 

\paragraph{Key Challenges} 
Scalable Heterogeneous Computing (SHC) platforms are gaining popularity, but it is becoming more and more complex to program these systems effectively and to evaluate their performance at scale. Performance engineering of applications must take into account multi-layered language and runtime systems, while mapping low-level actions to high-level programming abstractions.  Runtime systems such as Kokkos can shield the complexities of programming SHC systems from the programmers, but pose challenges to performance evaluation tools.  Better integration of performance technology is required.  Exposing parallelism to compilers using higher level constructs in the intermediate language provides additional opportunities for instrumentation and mapping of performance data.  It also makes possible developing new capabilities for observing multiple layers of memory hierarchy and I/O subsystems, especially for NVM-based HPC systems. 

\paragraph{Solution Strategy} Compilers and runtime systems can expose several opportunities for performance instrumentation tools such as TAU.  For instance, using the OpenACC profiling interface, TAU can tap into a wealth of information during kernel execution on accelerators as well measure data transfers between the host and devices. This can highlight when and where these data transfers occur and how long they last.  By implementing compiler-based instrumentation of LLVM compilers with TAU, it is possible to how the precise exclusive and inclusive duration of routines for programs written in C, C++, and Fortran.  Furthermore, we an take advantage of the Kokkos profiling interface to help map lower level performance data to higher level Kokkos constructs that are relevant to programmers. The instrumentation at the runtime system level can be achieved by transparently injecting the TAU Dynamic Shared Object (DSO) in the address space of the executing application. This requires no modification to the application source code or the executable. 

\begin{figure}[htb]
\centering
\includegraphics[width=5in]{projects/2.3.2-Tools/2.3.2.09-PROTEAS/tau-3d}
\caption{
  TAU's ParaProf profile browser shows the parallel performance of the AORSA2D application.
}
\label{figure:tau}
\end{figure}

\paragraph{Recent Progress}
\begin{enumerate}
\item \textbf{Compiler-based instrumentation:} Added support for compiler-based instrumentation in TAU for LLVM Clang and Flang compilers on IBM Power 9, Intel x86\_64, and Cray XC systems. 

\item \textbf{Kokkos} Added support for the Kokkos profiling interface in TAU by extending the tau\_exec tool that preloads the TAU DSO in an uninstrumented binary and applied it to ECP benchmarks.

\item \textbf{CANDLE} Extended TAU to support performance evaluation of Python and CUDA and applied it to evaluate the performance of the CANDLE ECP Benchmarks on IBM Power and Cray XC systems. 

\item \textbf{Improved CUDA and OpenMP support} Added support for newer GPUs and enhancements to the CUPTI profiling interface in TAU. Updated the OpenMP Tools Interface support in TAU. 
\end{enumerate}

\paragraph{Next Steps}
\begin{enumerate}
\item \textbf{NVM instrumentation} 
Design and implement support for supporting deep memory hierarchies in TAU for supporting MCDRAM based systems. This includes support for memkind hbm-malloc and Fortran FASTMEM directives. 

\item \textbf{PHIRE} 
Design and implement support for instrumentation of higher level PHIRE intermediate language constructs in TAU. 

\item \textbf{Outreach}
Outreach activities to demonstrate comprehensive performance evaluation support in TAU for OpenARC, LLVM, CUDA, Kokkos, and NVM based programming frameworks for SHC platforms. 
\end{enumerate}