\subsubsection{\stid{3.01} xSDK} 
\paragraph{Overview} The xSDK project is creating a value-added aggregation of DOE math and scientific libraries through the xSDK~\cite{xsdk:homepage}, which increases the combined usability, standardization, and interoperability of these libraries as needed by ECP. The project focuses on community development and a commitment to combined success via quality improvement policies, better build infrastructure, and the ability to use diverse, independently developed xSDK libraries in combination to solve large-scale multiphysics and multiscale problems.  We are extending xSDK package community policies and developing interoperability layers among numerical libraries in order to improve code quality, access, usability, interoperability, and sustainability. Focus areas are (1) coordinated use of on-node resources, (2) integrated execution (control inversion and adaptive execution strategies), and (3) coordinated and sustainable documentation, testing, packaging, and deployment. %In FY20, the project also began to investigate and deploy multiprecision functionality in the ECP ST ecosystem to enable the use of low-precision hardware function units, reduce the pressure on memory and communication interfaces, and achieve improved performance.

xSDK is needed for ECP because it enables applications such as ExaAM and ExaWind to seamlessly leverage the entire scientific libraries ecosystem.  For example, ExaWind has extremely challenging linear solver scaling problems.  xSDK provides access to all scalable linear solvers with minimal changes.  xSDK is also an essential element of the product release process for ECP ST.  xSDK provides an aggregate build and install capability for all ECP math libraries that supports hierarchical, modular installation of ECP software.  Finally, xSDK provides a forum for collaborative math library development, helping independent teams to accelerate adoption of best practices, enabling interoperability of independently developed libraries and improving developer productivity and sustainability of the ECP ST software products.

\paragraph{Key Challenges}
The complexity of application codes is steadily increasing due to more sophisticated scientific models.  While some application areas will use Exascale platforms for higher fidelity, many are using the extra computing capability for increased coupling of scales and physics.  Without coordination, this situation  leads to difficulties when building application codes that use 8 or 10 different libraries, which in turn might require additional libraries or even different versions of the same libraries.

The xSDK represents a different approach to coordinating library development and deployment.  Prior to the xSDK, scientific software packages were cohesive with a single team effort, but not across these efforts. The xSDK goes a step further by developing community policies followed by each independent library included in the xSDK.  This policy-driven, coordinated approach enables independent development that still results in compatible and composable capabilities.

\paragraph{Solution Strategy}

The xSDK effort has two primary thrusts:
\begin{enumerate}
	\item \textbf{Increased interoperability:} xSDK packages can be built with a single Spack package target.  Furthermore, services from one package are accessible to another package.
	\item \textbf{Increased use of common best practices:}  The xSDK has a collection of community policies that set expectations for a package, from best design practices to common look-and-feel.
\end{enumerate}

xSDK interoperability efforts began first with eliminating incompatibilities that prohibited correct compilation and integration of the independently developed libraries.  These issues include being able to use a common version of a library by another library.  The second, and ongoing phase is increased use of one package's capabilities from another. xSDK community package policies~\cite{xsdk-policies:homepage,xsdk-policies:github} are a set of minimum requirements (including topics of configuring, installing, testing, MPI usage, portability, contact and version information, open source licensing, namespacing, and repository access) that a software package must satisfy in order to be considered xSDK compatible. The designation of xSDK compatibility informs potential users that a package can be easily used with others and makes configuration and installation of xSDK software and other HPC packages as efficient as possible on common platforms, including standard Linux distributions and Mac OS X, as well as on target machines currently available at DOE computing facilities (ALCF, NERSC, and OLCF) and eventually on new Exascale platforms.
Community policies for the xSDK promote long-term sustainability and interoperability among packages, as a foundation for supporting complex multiphysics and multiscale ECP applications. In addition, because new xSDK packages will follow the same standard, installation software and package managers (for example, Spack~\cite{gamblin+:sc15}) can easily be extended to install many packages automatically.

For the adaptive execution effort, the team is working toward GPTune, a Gaussian process tuner, to help math library users find the optimal parameter settings for the libraries to achieve high performance for their applications. In addition, an interface will be created to also give access to alternate autotuners.

%For the multiprecision effort, the project assesses current status and functionalities, advances the theoretical knowledge on multiprecision algorithms, design prototype implementations and multiprecision interoperability layers, deploys production-ready multiprecision algorithms in the xSDK math libraries, ensures multiprecision cross-library interoperability and integrates multiprecision algorithms into ECP application projects.

\paragraph{Recent Progress}

The xSDK team developed a suite of example codes that demonstrate interoperabilities between select xSDK libraries, xsdk-examples v.0.1.0~\cite{xsdk-examples}. The suite includes a build system and documentation in the subfolders of the codes and can be built with Spack~\cite{gamblin+:sc15}. It provides training for xSDK users on mixed package use. It also serves as test suite and will be included in testing of future xSDK releases.
Figure~\ref{fig:xsdk-schematic} illustrates the xSDK libraries and their interoperabilities represented in the first release.

The xSDK team also released version v.0.6.0 of the xSDK community policies~\cite{xsdk-policies:github}. It includes a new recommended policy on documentation quality. Since the switch from the original xSDK installer to Spack as the xSDK package installer has facilitated the build of the xSDK, the team could simplify policy M1 by merging it with M16 and abandoning the installation policies. In place of the installation policies, Spack variant guidelines have been provided, and a new policy M16 was created to keep the installation policy requirement that xSDK libraries need to have an option to be configured in debug mode.

\begin{figure}[htb]
	\centering
	\includegraphics[width=4.2in]{projects/2.3.3-MathLibs/2.3.3.01-xSDK/xSDK-examples-diagram.png}
	%\includegraphics[width=4.2in]{xSDK-examples-diagram.png}
	\caption{\label{fig:xsdk-schematic} xSDK packages and interoperabilities represented in version v.0.1.0 of the xsdk-examples test suite. A$\rightarrow$B indicates that A uses functionalities of B}
\end{figure}


The first version of the GPTune autotuning software for parameter optimization of HPC codes was released~\cite{gptune:homepage}. It was evaluated by tuning several ECP math libraries and applications codes using up to 2048 Cori Haswell cores. GPTune achieved a performance gain of up to 60 percent compared to default parameter settings. It outperformed two state-of-the-art tuners, OpenTuner and HpBandster, up to 2.5, when tuning ScaLAPACK QR.

%The xSDK team performed a literature survey on mixed precision algorithms and summarized the results in a report and a paper submitted to IJHPC~\cite{Anztetal2020}. It designed an accessor that separates memory precision from arithmetic precision and provided a document that contains details about the design.

\paragraph{Next Steps}

Our next efforts include 
\begin{itemize}
    \item a new xSDK release with two additional math libraries heFFTe and SLATE, 
    \item development of new interoperabilities between xSDK libraries and their inclusion in xsdk-examples, 
   % \item advancing multiprecision capabilities for solvers, preconditioners, and other ECP-relevant kernels in xSDK libraries,
  %  \item a production-ready implementation of an accessor that separates memory precision from arithmetic precision, and an example to showcase its use,
    \item enhancing GPTune with new features, such as transfer learning, incorporation of predictive models, and speeding up the internal Gaussian process modeling algorithms,
    \item design and implementation of a code quality toolkit that automates analyses and activities related to code testing, documentation, and use.
    
\end{itemize}

\newpage
%\begin{enumerate}
%	\item \textbf{Include more libraries:} xSDK4ECP will continue efforts to expand the number of participating packages, adapt community policies, and exploit increased interoperability.  We are coordinating with broader SDK efforts and working toward the inclusion of additional domain application packages.
%	\item \textbf{Extend application usage:}  xSDK4ECP will continue partnering with application teams to evaluate the effectivness of current functionality and to motivate new capabilities.
%	\item \textbf{Autotuning of code performance:} The team will finish a prototype autotuning framework using the multi-output Gaussian Process ML approach which includes multitask and transfer learning.
%	\item \textbf{Multiprecision effort:} The xSDK4ECP team will investigate the use of multiprecision for the math libraries and document the results of the investigation. 

% 		\item \textbf{Process control transfer interfaces:} The ever-increasing use of concurrency within the top-level MPI processes requires that computational resources used by an application or library can be transferred to another library. Transfer of these resources is essential for obtaining good performance.  The xSDK project will develop interfaces to support sharing and transfer of computational resources.	
%\end{enumerate}
