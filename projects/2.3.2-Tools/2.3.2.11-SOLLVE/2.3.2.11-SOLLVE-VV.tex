\subsubsection{\stid{2.11}Validation and Verification Testsuite}

\paragraph{Overview}
The OpenMP Validation and Verification Testsuite (OMPVV) presents the community with a test suite comprised of functional tests and application kernels that are written with the intention of verifying usability of OpenMP features. The test suite aims to accomplish several goals of equal importance. The first goal is to provide vendors with a straightforward way to examine the status of their implementations. The next goal is to support application developers in identifying whether their system can support the specific OpenMP features they wish to utilize. Both of these are accomplished by running the full test suite and generating a report summary, which provides the user with a simple pass-fail report for OpenMP features. These insights also include where the tests are failing (compile time or runtime) and what is the error encountered. Our full suite of OpenMP tests is made publicly available on Github \cite{sollvevvgithub} and full documentation in available the corresponding website \cite{sollvevvwebsite}. 

\paragraph{Key Challenges}
Every few years, the OpenMP Architecture Review Board releases new versions of their specification. These new releases of the specification, 5.0 in November of 2018, 5.1 in November of 2020, and 5.1 in November of 2021 introduce new clauses and directives that many application developers and general users are keen on utilizing. Thus, when the updates are published, we set out to create tests for each new feature or new directive. It takes time for the compiler developers to develop implementations for several of the new features, as a result there is a gap between versions released by the specification, implementations developed and implementations made available for the application developers. Due to this fact, we are often times writing test cases for new clauses and directives even before the implementations begin to exist. This can be quite a challenge since we cannot even compile or execute them right away, so would need to revisit the test cases as soon as the implementations are made available. 

\paragraph{Solution Strategy}
 For every new specification version we begin by classifying the priority of each new feature based on input from ECP and (sometimes) CAAR application teams. We also track LLVM's OpenMP implementation status and formulate a priority list for test implementation. As more features are implemented by LLVM, we continue to create functional tests cases and verify our own work by running the tests on several heterogeneous systems that represent multiple vendors and OpenMP compiler implementations. In previous years, we followed the same strategy for providing OpenMP 4.5 test coverage \cite{vandv2019}.

\paragraph{Recent Progress}
Over the past year, we have been able to provide tests (C, C++) that cover almost all of the new features introduced in OpenMP 5.0. We continue to improve our Fortran tests coverage for 5.0 features as well. We periodically report the pass-fail status and individual test results on state-of-the-art systems such as Oak Ridge National Lab's Summit and Spock computing systems as well as NERSC's Cori system \cite{sollvevvwebsite}. This provides a useful tracking system for vendors and application developers alike. Also, we have created functional tests for several of the latest OpenMP 5.1 features, namely: atomic compare, C++ attribute specifier, declare variant, default first private, tiling etc. These have been implemented by LLVM and deemed of high importance to AD teams. 

\paragraph{Next Steps}
We will continue C/C++ test development for OpenMP 5.1 specification while we maintain, improve and fix any issues discovered over our previously developed test. We have continued to follow the same strategy as we did for providing test coverage of 5.0 features. We periodically run our tests on the newest available compiler implementations that support OpenMP and maintain the results \cite{sollvevvwebsite}. Additionally, we are tracking AD teams use of OpenMP features and incorporating application kernels in our tests that are outside the scope of individual feature tests but test a legal combination of OpenMP features that is critical for the application.