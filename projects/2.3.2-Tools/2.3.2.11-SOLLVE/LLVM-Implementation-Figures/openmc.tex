
  \definecolor{col0o}{RGB}{0,146,146}
  \colorlet{col0}{col0o!80}
  \definecolor{col1o}{RGB}{182,109,255}
  \colorlet{col1}{col1o!70}
  \definecolor{col2}{RGB}{219,209,0}
  \definecolor{col3}{RGB}{255,182,119}
  \definecolor{col8}{RGB}{182,255,219}
  \definecolor{col5}{RGB}{36,255,36}
  \definecolor{col6}{RGB}{182,219,255}
  \definecolor{col7}{RGB}{255,109,182}
  \definecolor{col18}{RGB}{255,255,109}
  \definecolor{col9}{RGB}{73,0,146}
  \definecolor{col10}{RGB}{146,73,0}
  \definecolor{col11}{RGB}{146,0,0}
  \definecolor{col12}{RGB}{0,109,219}
  \definecolor{col13}{RGB}{0,73,73}
  \definecolor{col14}{RGB}{73,0,73}
  \definecolor{col15}{RGB}{73,73,0}
  \definecolor{col16o}{RGB}{255,36,36}
  \colorlet{col16}{col16o!80}
  \definecolor{col17o}{RGB}{36,36,255}
  \colorlet{col17}{col17o!50}
  \definecolor{col4}{RGB}{255,182,219}
  
  \definecolor{Maroon}{HTML}{990000} 

\begin{figure}[h!]
  \vspace*{3mm}
  \centering

\resizebox{\linewidth}{!}{%
  %\includegraphics{includes/openmc.pdf}

\begin{tikzpicture}
\begin{axis}[
    compat=newest,
    enlarge y limits={rel=0.20},
    enlarge x limits={abs=0.5},
    ybar,
    ymode = log,
    log basis y = 10,
    bar width=40,
    xtick={1,2,3,4,5},
    xticklabels={CPU\\32 cores, A100 initial\\ performance, A100 after application-\\compiler co-design\\ (OpenMC + LLVM), A100 after further\\ algorithmic changes\\ (OpenMC team), AMD MI100 perf-\\ormance latest\\ OpenMC + LLVM},
    xtick style={draw=none},
    xticklabel style={align=center,font=\small},
    %xmajorticks=false,
    ymajorgrids=true,
    ylabel={particles/sec},
    %ylabel style = {align=center,at={(-0.08,0.5)}},
    %yminorgrids=true,
    grid style=dashed,
    height=7cm,
    width=18cm,
    visualization depends on ={y \as \y},
    visualization depends on ={rawy \as \rawy},
    %log ticks with fixed point,
    every node near coord/.append style = {
      %name=a\coordindex,
      %font=\small,
      /pgf/number format/fixed,
      %fill=white,
      %yshift=-2pt,
      %inner sep=0pt,
    },
    every axis plot/.append style={
      bar shift=0pt,
    },
  ]
\addplot[
    %nodes near coords,
    %point meta=rawy,
    nodes near coords={%
      \begingroup%
        % this group is merely to switch to FPU locally.
        % Might be unnecessary, but who knows.
        \pgfkeys{/pgf/fpu}%
        \pgfmathparse{\pgfplotspointmeta<4}%
        \global\let\issmall=\pgfmathresult%
        \pgfmathparse{\pgfplotspointmeta==0}%
        \global\let\iszero=\pgfmathresult%
        \pgfmathparse{\pgfplotspointmeta==1}%
        \global\let\isone=\pgfmathresult%
      \endgroup%
      \pgfmathfloatcreate{0}{0.0}{0}%
      \let\ZERO=\pgfmathresult%
      \pgfmathfloatcreate{1}{1.0}{0}%
      \let\ONE=\pgfmathresult%
      \color{black}%
      \ifx\iszero\ONE%
      {%
        \hspace*{-5mm}%
        \global\let\ycoord=\ZERO%
        \textbf{OoM}%
      }%
      \else%
        \ifx\isone\ONE%
        {%
          \textbf{base}
          \global\let\ycoord=\ZERO%
        }%
        \else%
          \ifx\issmall\ONE%
          {%
            %\global\let\ycoord=\ZERO
            \textbf{\phantom{,}\pgfmathprintnumber[precision=2]{\rawy}\phantom{,}}%
          }%
          \else%
            %\hspace*{-12mm}
            %\hspace*{-4mm}
            %\global\let\ycoord=\rawy
            %\colorbox{white}{%
              \textbf{\pgfmathprintnumber[precision=2]{\rawy}}%
            %}
          \fi%
        \fi%
      \fi%
    },
    every node near coord/.append style = {
        name=a\coordindex,
        /pgf/number format/fixed,
        yshift=0.5pt,
        fill=white,
        inner sep=0pt,
    },
    fill=Maroon,
  ]
  coordinates {
(2,602)
(3,58529)
(4,349685)
};

\addplot[
    %nodes near coords,
    %point meta=rawy,
    nodes near coords={%
      \begingroup%
        % this group is merely to switch to FPU locally.
        % Might be unnecessary, but who knows.
        \pgfkeys{/pgf/fpu}%
        \pgfmathparse{\pgfplotspointmeta<4}%
        \global\let\issmall=\pgfmathresult%
        \pgfmathparse{\pgfplotspointmeta==0}%
        \global\let\iszero=\pgfmathresult%
        \pgfmathparse{\pgfplotspointmeta==1}%
        \global\let\isone=\pgfmathresult%
      \endgroup%
      \pgfmathfloatcreate{0}{0.0}{0}%
      \let\ZERO=\pgfmathresult%
      \pgfmathfloatcreate{1}{1.0}{0}%
      \let\ONE=\pgfmathresult%
      \color{black}%
      \ifx\iszero\ONE%
      {%
        \hspace*{-5mm}%
        \global\let\ycoord=\ZERO%
        \textbf{OoM}%
      }%
      \else%
        \ifx\isone\ONE%
        {%
          \textbf{base}
          \global\let\ycoord=\ZERO%
        }%
        \else%
          \ifx\issmall\ONE%
          {%
            %\global\let\ycoord=\ZERO
            \textbf{\phantom{,}\pgfmathprintnumber[precision=2]{\rawy}\phantom{,}}%
          }%
          \else%
            %\hspace*{-12mm}
            %\hspace*{-4mm}
            %\global\let\ycoord=\rawy
            %\colorbox{white}{%
              \textbf{\pgfmathprintnumber[precision=2]{\rawy}}%
            %}
          \fi%
        \fi%
      \fi%
    },
    every node near coord/.append style={
        /pgf/number format/fixed,
        yshift=0.5pt,
        fill=white,
        inner sep=0pt,
    },
    fill=col6,
  ]
  coordinates {
(1,54913)
};
\addplot[
    %nodes near coords,
    %point meta=rawy,
    nodes near coords={%
      \begingroup%
        % this group is merely to switch to FPU locally.
        % Might be unnecessary, but who knows.
        \pgfkeys{/pgf/fpu}%
        \pgfmathparse{\pgfplotspointmeta<4}%
        \global\let\issmall=\pgfmathresult%
        \pgfmathparse{\pgfplotspointmeta==0}%
        \global\let\iszero=\pgfmathresult%
        \pgfmathparse{\pgfplotspointmeta==1}%
        \global\let\isone=\pgfmathresult%
      \endgroup%
      \pgfmathfloatcreate{0}{0.0}{0}%
      \let\ZERO=\pgfmathresult%
      \pgfmathfloatcreate{1}{1.0}{0}%
      \let\ONE=\pgfmathresult%
      \color{black}%
      \ifx\iszero\ONE%
      {%
        \hspace*{-5mm}%
        \global\let\ycoord=\ZERO%
        \textbf{OoM}%
      }%
      \else%
        \ifx\isone\ONE%
        {%
          \textbf{base}
          \global\let\ycoord=\ZERO%
        }%
        \else%
          \ifx\issmall\ONE%
          {%
            %\global\let\ycoord=\ZERO
            \textbf{\phantom{,}\pgfmathprintnumber[precision=2]{\rawy}\phantom{,}}%
          }%
          \else%
            %\hspace*{-12mm}
            %\hspace*{-4mm}
            %\global\let\ycoord=\rawy
            %\colorbox{white}{%
              \textbf{\pgfmathprintnumber[precision=2]{\rawy}}%
            %}
          \fi%
        \fi%
      \fi%
    },
    every node near coord/.append style={
        /pgf/number format/fixed,
        yshift=0.5pt,
        fill=white,
        inner sep=0pt,
    },
    fill=col3,
  ]
  coordinates {
  (5,93918)
};
\end{axis}

\draw[ultra thick,->,shorten <=1mm,shorten >=3mm] ($(a0.south east) + (3mm,0mm)$) -- node[midway,above,rotate=55,fill=white,inner sep=1pt,yshift=2pt] {97x} ($(a1.south west) + (-0.5mm,1mm)$);
\draw[ultra thick,->,shorten <=1mm,shorten >=3mm] ($(a1.south east) + (3mm,0mm)$) -- node[midway,above,rotate=30,fill=white,inner sep=1pt,yshift=2pt] {6x} (a2.south west);

\end{tikzpicture}
}

\vspace*{-2mm}

\caption{
  Performance results for the OpenMC~\cite{romano2013openmc} OpenMP offloading code that simulates transport of neutral particles using the Monte Carlo methodology.
  The application is part of the ExaSMR ECP project and known for its proxy applications (XSBench~\cite{XS_Tramm_2014} and RSBench~\cite{RS_Tramm_2014}).
  Results show the almost 100x speedup achieved by co-optimizing the full OpenMC application with the LLVM compiler for OpenMP offloading on NVIDIA A100 GPUs.
  In the process, changes to both OpenMC and LLVM were made, and the latter were often triggered via command line options or assumptions.
  Guided by compiler remarks, other applications could benefit similarly and with far less involvement from a compiler developer---but only if the remarks, and later the suggested code changes, are clear, actionable, and effective.
 For context, we also present the AMD MI100 numbers obtained with the latest versions
 of OpenMC and the LLVM/Clang compiler.
 Since the LLVM/OpenMP offloading backend for AMD GPU is new and only recently became stable, we expect substantial performance improvements as we start our tuning efforts.
 Even without, we believe the results shows how performance portability is certainly
 within reach for a full application using the LLVM/OpenMP offloading environment. \\
  The data for the figure was generated and generously provided by John Tramm.
}

  \label{fig:openmc}
  %\vspace*{-2mm}
  \vspace*{2mm}
\end{figure}
