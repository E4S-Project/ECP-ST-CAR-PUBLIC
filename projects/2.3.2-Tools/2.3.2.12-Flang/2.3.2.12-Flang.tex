\subsubsection{\stid{2.12} Flang}\label{subsubsect:flang}

\paragraph{Overview}

The Flang project provides an open-source Fortran standard 
\cite{iso-fortran-2004, iso-fortran-2010, iso-fortran-2018}
compiler front-end for the LLVM Compiler Infrastructure (see
\url{http://llvm.org})~\cite{llvm:homepage}.  Flang was formally
accepted as an official component of LLVM in 2019 and merged portions
of its initial code base into the main LLVM repository in April 2020. 
Work continues today with a growing set of contributors to the
code base.  Leveraging LLVM, Flang will provide a
cross-platform Fortran solution available to ECP and the broader
international LLVM community. The goals of the project include
extending support to GPU accelerators and target Exascale systems, and
supporting LLVM-based software and tools of interest to a large
deployed base of Fortran applications.

LLVM's growing popularity and wide adoption make it an integral part
of the modern software ecosystem. Flang will provide a foundation
for Fortran that will complement and interoperate with the
Clang C/C++ compiler and other tools within the LLVM infrastructure.
We aim to provide a modern, open-source Fortran implementation that is
stable, has an active footprint within the LLVM community, and will
meet the specific needs of ECP as well as the broader scientific
computing community.

\paragraph{Key Challenges}
Today there are several commercially-supported Fortran compilers,
typically available from one vendor and often for a limited set of
platforms.  None of these are open source.  While the GNU gfortran compiler is
open source and available on a wide variety of platforms, the source base is not modern
LLVM-style C++ and the GPL license is not compatible with
LLVM.  This places limits on a wide breath of potential collaborations, and thus has an
impact on broader community participation and adoption.

The primary challenge of this project is to create a source base with
the maturity, features, and performance of proprietary solutions with
the cross-platform capability of GNU compilers, and which is licensed
and coded in a style that will be embraced by the LLVM community. 
Additional challenges come from supporting all Fortran
language features, language extensions (e.g., OpenMP, OpenACC), and
scalability required for effective use of exascale-class systems. 

\paragraph{Solution Strategy}

With the adoption of Flang into the LLVM community, our strategy
focuses on establishing and growing a strong community around it and the development
and delivery of a solid, alternative Fortran compiler for DOE's
platforms.  It is critical that we be good shepherds within
the LLVM community to successfully establish this community for Flang.
This external engagement is in the best interest
of ECP as well as the long-term success of Fortran, and enables leveraging
the significant momentum and strengths of this widely used and accepted
code base. 

Our path to success will rely on significant testing across not only
the various facilities but also across a very broad and diverse set of
applications. Given the early development stage of Flang, this testing
will be paramount in the delivery of a robust infrastructure to ECP
and the broader community.

\paragraph{Recent Progress}

After several years of effort and support from NNSA, Flang was 
successfully adopted by the LLVM community and has transitioned 
from a stand-alone git repository to one hosted by the main LLVM project.  This 
represents a significant result and the community reviewed and accepted code is
available via GitHub:
%
%
\url{https://github.com/llvm/llvm-project/}.

The current capabilities of Flang include the parsing and semantic
analysis of the full Fortran 2018 standard and OpenMP 5.x. As part of the
development of the parsing and semantic analysis portions of the
front-end, over five million lines of Fortran code have been
successfully processed. Beyond parsing and semantics, we have been
focusing our efforts on creating a Fortran-centric intermediate
representation (Fortran IR [FIR]) that leverages recent activities
within Google on 
\href{Multi-Level Intermediate Representations} 
{https://www.blog.google/technology/ai/mlir-accelerating-ai-open-source-infrastructure/}
(MLIR) for use with the implementation of FIR.
With the development of FIR progressing, we have completed
the first full (sequential) compiler with the full set of F77 
capabilities, and will soon complete the full set of F95 capabilities.
Implementations of the later standards will follow in chronological order.

\paragraph{Next Steps}
Our short-term priorities are focused on up-streaming of the
sequential compiler with Fortran 95 support, completing implementation of the remaining Fortran standards,
the creation of a significant testing infrastructure, and helping to play a key role in
the interactions and overall discussions within the LLVM community.  Longer term efforts
will shift to support OpenMP 5.x features critical to ECP applications
on the target Exascale platforms.  We are actively exploring finding a
common leverage point between Clang's current OpenMP code base and
Flang.  This would enable the reuse of existing code versus writing
everything from scratch in Flang.  We see this as a critical path
forward to enabling a timely release of a node-level parallelizing
compiler for ECP.  Additional work will focus on features that would
benefit Fortran within the LLVM infrastructure as well as general and
targeted optimization and analysis capabilities.

