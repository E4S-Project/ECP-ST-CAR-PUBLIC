\subsubsection{\stid{5.04} Sandia ATDM Software Ecosystem and Delivery-- OS/On-Node Runtime} 

\paragraph{Overview} 

This project is part of the NNSA/ASC program and is primarily focused
on operating system and runtime system (OS/R) technology
development and evaluation. This project is tightly aligned with the
Qthreads ECP project.

This project focuses on the design, implementation, and evaluation of
OS/R interfaces, mechanisms, and policies supporting the efficient
execution of the ATDM application codes on next-generation ASC
platforms. Priorities in this area include the development of
lightweight tasking techniques that integrate network communication,
interfaces between the runtime and OS for management of critical
resources (including multi-level memory, non-volatile memory, and
network interfaces), portable interfaces for managing power and
energy, and resource isolation strategies at the operating system
level that maintain scalability and performance while providing a more
full-featured set of system services. The OS/R technologies developed
by this project will be evaluated in the context of ATDM application
codes running at large-scale on ASC platforms. Through close
collaboration with vendors and the broader community, the intention is
to drive the technologies developed by this project into
vendor-supported system software stacks and gain wide adoption
throughout the HPC community.

\paragraph{Key  Challenges}

Key challenges for this project include:

\begin{itemize}

  \item {\bf Improve the understanding of the use of containers and
    virtualization technology to support ATDM applications and
    workloads} Containers are gaining popularity as a way to package
  applications and virtualize the underlying OS to allow a set of
  executables built for one platform to be run unmodified on a
  different platform. There are several different approaches to
  building and deploying containers, each with differing sets of
  capabilities and features.

  \item {\bf Characterizing applications use of MPI and sensivity to
      system noise}  Understanding how applications use MPI and its
    associated  network resources requires both application- and
    hardware-level information that must be coordinated on time scales
    of less than a microsecond. It is also extremely difficult to isolate
    the sources of system noise and characterize the non-local side
    effects of unplanned detours that interrupt application execution
    flow.

\end{itemize}


\paragraph{Solution Strategy}

The strategy for containers and virtualization is to evaluate the
different technology options using ATDM applications and workflows and
compare the results against a set of evaluation criteria.

In order to characterize applications use of MPI and sensitivity to
system noise, this project has developed a simulation environment that
can be used to track MPI and network resource usage. This project is
also using lightweight operating systems, which are virtually devoid
of system noise, help understand how applications, especially those
employing an ATM programming model, are impacted by OS noise.

\paragraph{Recent Progress}

Potential use cases and technical options for container technologies
in ATDM were evaluated and published in a conference
paper~\cite{Younge:Tale:2017}, which was nominated for best paper.

A journal article with analyses of MPI queue behavior observed during
executions of Sandia mini-apps, as well as LAMMPS and CTH, has been
accepted for publication~\cite{Ferreira:Characterizing:2018}.  The
techniques were also applied to SPARC, and an expanded tech report
version including the SPARC results will be available soon.

\paragraph{Next Steps}

Sandia's Kitten lightweight kernel has been ported to the Cray
XC architecture and has been demonstrated on the Volta testbed system.
Going forward, we are working on plans to port Kitten and/or the RIKEN
McKernel lightweight kernel to the Vanguard Petascale ARM platform.
We will be visiting RIKEN in early March to discuss plans and
collaboration opportunities.

Close to having the Qthreads backend to Kokkos patched up after last
year's frontend-to-backend API changes.  Completion is contingent on
resolution of the Kokkos team's remaining patches to the threads
back-end, with which Qthreads shares common hooks.

We are participating in an ECP working group on container technology
and using the results of evaluation to guide future activities in this
area.


