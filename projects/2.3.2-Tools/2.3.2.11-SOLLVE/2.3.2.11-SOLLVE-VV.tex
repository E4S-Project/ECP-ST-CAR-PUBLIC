\subsubsection{\stid{2.11}Validation and Verification Testsuite}

\paragraph{Overview}
The OpenMP Validation and Verification Testsuite (OMPVV) presents the community with a test suite comprised of functional tests and application kernels that are written with the intention of verifying usability of OpenMP features. The test suite aims to accomplish several goals of equal importance. The first goal is to provide vendors with a straightforward way to examine the status of their implementations. The next goal is to support application developers in identifying whether their system can support the specific OpenMP features they wish to utilize. Both of these are accomplished by running the full test suite and generating a report summary, which provides the user with a simple pass-fail report for OpenMP features. These insights also include where the tests are failing (compile time or runtime) and what is the error encountered. Our full suite of OpenMP tests is made publicly available on Github \cite{sollvevvgithub} and full documentation in available the corresponding website \cite{sollvevvwebsite}. 

\paragraph{Key Challenges}
Every few years, the OpenMP Architecture Review Board releases new versions of their specification. These new releases of the specification, 5.0 in November of 2018 and 5.1 in November of 2020, introduce new clauses and directives that many application developers and general users are keen on utilizing. Thus, when the updates are published, we set out to create tests for each new feature or new directive. It takes time for the compiler developers to develop implementations for several of the new features, as a result there is a gap between versions released by the specification, implementations developed and implementations made available for the application developers. Due to this fact, we are often times writing test cases for new clauses and directives even before the implementations begin to exist. This can be quite a challenge since we cannot even compile or execute them right away, so would need to revisit the test cases as soon as the implementations are made available. 

\paragraph{Solution Strategy}
 When OpenMP 5.0 was released in 2018, we began by identifying all of the features reported to be 'implemented' or 'partially implemented' by LLVM's OpenMP implementation status page and placed those tests at the top of our priority list. Additionally, we identified the features that were of great importance to application developers and also associated these as 'high priority' tests. Since the release of OpenMP 5.1 in 2020, we have adopted the same strategy as before. As more features are implemented by LLVM, we continue to create functional tests cases and verify our own work by running the tests on several heterogeneous systems that represent multiple vendors and OpenMP compiler implementations. In previous years, we followed the same strategy for providing OpenMP 4.5 test coverage \cite{vandv2019}.

\paragraph{Recent Progress}
Over the past year, we have been able to provide tests (C, C++, and Fortran90) that cover almost all of the new features introduced in OpenMP 5.0. We have continued to report the pass-fail status and individual test results on state-of-the-art systems such as Oak Ridge National Lab's Summit and Spock computing systems as well as NERSC's Cori system. Also, we have created functional tests for several of the latest OpenMP 5.1 features that have been implemented by LLVM and deemed of high importance to application developers.

\paragraph{Next Steps}
Following the release of the OpenMP 5.1 specification, we have continued to follow the same strategy as we did for providing test coverage of 5.0 features. As LLVM continues to make progress implementing the new 5.1 features, we will continue to generate new functional tests and run our tests on the newest available compiler implementations that support OpenMP.  Additionally, we have a list of features that application developers are interested in using and will consult this list as a priority list alongside the LLVM implementation status list. 