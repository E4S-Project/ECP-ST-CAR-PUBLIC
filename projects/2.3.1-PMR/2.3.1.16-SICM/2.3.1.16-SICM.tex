\subsubsection{\stid{1.16} SICM} 
\paragraph{Overview} The goal of this project is to create a universal interface for discovering, managing and sharing within complex memory hierarchies. The result will be a memory API and a software library which implements it. These will allow operating system, runtime and application developers and vendors to access emerging memory technologies. The impact of the project will be immediate and potentially wide reaching, as developers in all areas are struggling to add support for the new memory technologies, each of which offers their own programming interface. The problem we are addressing is how to program the deluge of existing and emerging complex memory technologies on HPC systems. This includes the MCDRAM (on Intel Knights Landing), NV-DIMM, PCI-E NVM, SATA NVM, 3D stacked memory, PCM, memristor, and 3Dxpoint. Also, near node technologies, such as PCI-switch accessible memory or network attached memories, have been proposed in exascale memory designs. Current practice depends on ad hoc solutions rather than a uniform API that provides the needed specificity and portability. This approach is already insufficient and future memory technologies will only exacerbate the problem by adding additional proprietary APIs. Our solution is to provide a unified two-tier node-level complex memory API. The target users for the low-level interface are system and runtime developers, as well as expert application developers that prefer full control of what memory types the application is using. The high-level interface is designed for application developers who would rather define coarser-level constraints on the types of memories the application needs and leave out the details of the memory management. The low-level interface is primarily an engineering and implementation project. The solution it provides is urgently needed by the HPC community; as developers work independently to support these novel memory technologies, time and effort is wasted on redundant solutions and overlapping implementations. We can achieve success due to our team’s extensive experience with runtimes and applications. Our willingness to work with and accept feedback from multiple hardware vendors and ECP developers differentiates our project from existing solutions and will ultimately determine the scale of adoption and deployment. 
\begin{itemize}
\item  Low-Level Interface: Finished refactor of low-level interface supporting memory arenas on different memory types. Added initial for Global Arrays, and OMPI-X. Reviewing features need to fully support these runtimes.

\item Analysis: evaluating ACME using Gem5. We are currently resolving some compatibility issues between the ACME build environment and our Gem5 virtual machine. We now have traces from ES3M and several mini-apps.
\item Analysis and High-Level interface: New tool-chain  based on Intel PEBs instrumentation that analyzes memory use and suggests placement.
\item Cost Models: We have extracted a lot of experimental data related to application memory use and layout. This was done with full applications on hardware with the memory throttling-based emulation methodology.

\item Cost Models: Development of a tool, Mnemo, which provides for automated recommendations of capacity sizing of heterogeneous memories for object store workloads. Given a platform with specific configuration of different memories, and a (representative) workload, we can quickly extract some of the relevant memory usage metrics, and produce cost-benefit estimation curves as a function of different capacity allocations of the different types of memories. The output of Mnemo are estimates which give its users information to make informed decisions about capacity allocations. This can have practical use in shared/capacity platforms, or to rightsize capacity allocations to collocated workloads. 
\end{itemize}
\paragraph{Next Steps}
\begin{itemize}
\item  Low-Level Interface: Focus on support of runtimes and adding feature requested to support Global Arrays, OpenMP and MPI. Test with proxy applications for functionality and correctness. Investigate Linux kernel modifications for page migration in collaboration with ECP project Argo 2.3.5.05 and RIKEN research center in Japan, on-going. Verify support of emerging OpenMP standards.
\item Document needs ACME climate app hybrid memory analysis (with ORNL collaborators and related to UNITY SSIO ASCR project)
\item Understand capabilities of hwloc and netloc with respect to OMPI-X needs and work with managers of those libraries.
\end{itemize}


