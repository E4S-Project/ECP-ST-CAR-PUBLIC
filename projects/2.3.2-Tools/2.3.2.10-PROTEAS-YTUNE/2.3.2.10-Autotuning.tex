\subsubsection{\stid{2.10} PROTEAS-TUNE: Autotuning} 
\label{sec:PROTEAS_TUNE_AUTOTUNING}

\paragraph{Overview}

We are developing tools and an application development workflow that separates a high-level C/C++/FORTRAN implementation from an architecture-specific implementation (OpenMP, CUDA, etc.), optimization, and  tuning.   This  approach
will enable Exascale application  developers to express and  maintain a
single, portable implementation of their computation that is also legal code
that can be compiled and run by using standard tools.   The autotuning compiler
and search framework will transform the baseline code into a   collection of
highly-optimized implementations. This reduces the need for extensive manual tuning.
Both code transformation and autotuning are essential in ECP for providing
performance portability on Exascale platforms.  Due to significant architectural
differences in ECP platforms, attaining performance portability may  require
fundamentally different  implementations of software -- different strategies for
parallelization, loop order,  data layout, and exploiting SIMD/SIMT.  A key
concern of ECP is the high cost of developing  and maintaining
performance-portable applications for  diverse Exascale architectures, including
manycore CPUs and GPUs. 
Ideally Exascale application developers would express their
computation separate from   its mapping to hardware, while autotuning compilers can automate this mapping and achieve performance portability.

\paragraph{Key Challenges}
Autotuning has the potential to dramatically improve the performance portability of Petascale and Exascale applications.  To date, autotuning has been used primarily in high-performance applications through tunable libraries or previously tuned application code that is integrated directly into the application.
If autotuning is to be widely used in the HPC community,
support for autotuning must address the software engineering challenges, manage configuration overheads, and continue to demonstrate significant performance gains and portability across architectures.
In particular, tools that configure the application must be integrated into the application build process so that tuning can be reapplied as the application and target architectures evolve.

\paragraph{Solution Strategy}
We are developing pluggable software infrastructure that incorporates
autotuning at different levels: compiler optimization, runtime configuration of application-level parameters and system software.
To guarantee success in the ECP time frame, we are collaborating with
application teams, such as SuperLU and QMCPACK, to impact performance of their
codes and libraries.

The autotuning compiler strategy revolves CHiLL, which has the following distinguishing features:
(1) \textit{Composable transformation and code generation}, such
that the same tool can be applied
to multiple different application domains;
(2) \textit{Extensible to new domain-specific transformations} that can be represented as transformations on loop nest iteration spaces are also
composable with existing transformations;
(3) \textit{Optimization strategies and parameters exposed to autotuning:}
By exposing high-level expression
of the autotuning search space as transformation recipes, the compiler writer, an expert programmer or embedded DSL designer can directly \
express how to compose
 transformations that lead to different implementations.
A part of our efforts in ECP are to migrate these capabilities of CHiLL
into the Clang/LLVM open-source compiler, as well as provide lightweight
interfaces through Python, C++, and REST APIs/web services.

For example, we have developed a \textit{brick data layout library and code generator} for
stencil computations.
Recent trends in computer architecture that favor computation over data movement incentivize high-order methods.  Paradoxically, high-order codes can be challenging for compilers/optimization to attain high performance.  Bricks enable high performance and make fine-grained data reuse and memory access information known at compile time.  The SIMD code generation achieves performance portability
for high-order stencils for both CPUs with wide SIMD units (Intel Knights
Landing) and GPUs (NVIDIA Pascal).  Integration with autotuning attains
performance that is close to Roofline performance bound for both manycore CPU
and GPU architectures and demonstrates strong scaling by reducing on-node data movement in communication.


ytopt/SurF is a machine-learning-based search software package for autotuning that consists of sampling a small number of input parameter configurations, evaluating them, and progressively fitting a surrogate model over the input-output space until exhausting the user-defined time or the maximum number of evaluations. The package is built based on Bayesian Optimization that solves any optimization problem and is especially useful when the objective function is difficult to evaluate. It provides an interface that deals with unconstrained and constrained problems. The software is designed to operate in the master-worker computational paradigm, where one master node fits the surrogate model and generates promising input configurations and worker nodes perform the computationally expensive evaluations and return the outputs to the master node. The asynchronous aspect of the search allows the search to avoid waiting for all the evaluation results before proceeding to the next iteration. As soon as an evaluation is finished, the data is used to retrain the surrogate model, which is then used to bias the search toward the promising configurations.


\paragraph{Recent Progress}


We have pursued the following main activities this year:

\textit{Autotuning capability in LLVM:}
The key idea is to support the use of pragmas in the C++ source to guide transformations to be applied. These can include the types of transformation recipes used in CHiLL, but also parallelization directives for OpenMP and OpenACC that would interact with SOLLVE and PROTEAS. Our initial focus is the implementation of user/tool-directed optimizations in Polly, which is a polyhedral framework in LLVM with some similar features to CHiLL. Several existing open-source LLVM projects allowing for just-in-time (JIT) compilation of C++ code have been identified and are being evaluated for use with autotuning. 

We have successfully used loop transformations directives implemented in Clang (ref. Section \ref{sec:PROTEAS_TUNE_LLVM}) to drive a machine learning process with YtOpt. The correctness of these transformations are checked in the Clang compiler by Polly such that it is not possible to result in wrong results. The search strategies used include a Globally Greedy~\cite{kruse2020search}, random search, beam search and our latest algorithm a customized Monte-Carlo Tree Search~\cite{wu2021autotuning}.
In addition to the PolyBench benchmark suite, we applied the search over a tree-shaped configuration space to three ECP Proxy apps, namely SW4Lite, CoMD and minrAMR with promising results.


\vspace*{.1in}
\noindent
\textit{ytopt/SuRF Supporting Autotuning Search}

Compiler directives such as pragmas can help programmers to separate an algorithm's semantics from its optimization. Pragma directives for code transformations are useful for assisting program optimization and are already widely used in OpenMP. A prototype of user-directed loop transformations using Clang and Polly was implemented for the ECP SOLLVE project.
Polly is LLVM's polyhedral loop optimizer which makes it easy to apply specific transformations as directed by pragmas.
While the SOLLVE team is working on integrating the changes into the official LLVM repository, only few of the changes have been upstreamed yet.
The additional loop transformation directives supported are loop reversal (inverting the iteration order of a loop), loop interchange (permuating the order of nested loops), tiling, unroll(-and-jam), array packing (temporarily copying the data of a loop's working set into a new buffer) and thread parallelization. 
More importantly, it supports composing multiple loop nest transformation in arbitrary order. Vectorization is also supported by LLVM's dedicated loop vectorizer.
These pragmas are intended to make applying common loop optimization technique much easier and allow better separation of a code's semantics and its optimization.

We developed and extended ytopt autotuning framework that leverages Bayesian optimization to explore the parameter space search of LLVM Clang/Polly loop optimization pragmas. We integrated and compared four different supervised learning methods within Bayesian optimization and evaluated their effectiveness. We selected six of the most complex PolyBench benchmarks and apply the newly developed LLVM Clang/Polly loop optimization pragmas to the benchmarks to optimize them. We then used ytopt/SuRF framework to optimize the pragma parameters to improve their performance. The experimental results showed that our autotuning approach outperforms the other compiling methods to provide the smallest execution time for the benchmarks syr2k, 3mm, heat-3d, lu, and covariance with two large datasets in 200 code evaluations for effectively searching the parameter spaces with up to 170,368 different configurations. 

In a number of settings, the search space exposed by the transformation pragmas in LLVM LLVM/Polly’s composable loop optimization is a tree, wherein each node represents a specific combination of loop transformations that can be applied to the code resulting from the parent node's loop transformations. Consequently, Bayesian optimization within ytopt/SuRF will become ineffective. To that end, we have developed ytopt/MCTS, a search algorithm based on Monte Carlo tree search (MCTS). The algorithm consists of two phases: exploring loop transformations at different depths of the tree to identify promising regions in the tree search space and exploiting those regions by performing a local search. Moreover, a restart mechanism is used to avoid the MCTS getting trapped in a local solution. The best and worst solutions are transferred from the previous phases of the restarts to leverage the search history. We compared our approach with breadth-first, beam, global greedy, and random search methods using PolyBench kernels and ECP proxy applications. Experimental results showed that our MCTS algorithm finds pragma combinations with a speedup of 2.3x over Polly's heuristic optimizations on average.

CCS is a autotuning interface that we developed recently for ECP Kokkos library. CCS aims at providing interoperability between autotuning frameworks and applications with autotuning needs. It does so by providing a C interface with which users can describe their tuning problem, but also write their own tuners, possibly in higher-level languages. At the start of the collaboration with Kokkos team, CCS did not match up exactly with the Kokkos model of tuning. In CCS, there were no input types, and the concept of output types, objective space in CCS, was much more complex, providing a system of constraints among those types. It also had a much more complex view of a good run, having an objective space concept to specify how well a given configuration performed. Through codesign, CCS has adopted the input type concept as the features space, while maintaining the full expressiveness of their other concepts. This is progressing to the point where, as uses develop new tuners to the CCS infrastructure, they get the ability to tune Kokkos as well.



\vspace*{.1in}
\noindent
\textit{Brick Library:}
We developed a code generator for the Brick Data Layout library for stencils
that is performance-portable across CPU and GPU architectures, and addresses the
needs of modern multi-stencil and high-order stencil computations. The key
components of our approach that lead to performance portability are (1) a
fine-grained brick data layout designed to exploit the inherent multidimensional
spatial locality common to stencil computations; (2) vector code generation that
can either target wide SIMD CPU instructions sets such as AVX-512 and SIMT
threads on GPUs; and, (3) integration with autotuning framework to apply
architecture-specific tuning. For a range of stencil computations, we show that
it achieves high performance for both the Intel Knights Landing (Xeon Phi) CPU,
and the NVIDIA GPUs \cite{P3HPC_Bricks,zhao2019}. This year we extended the library in multiple ways.
We show that the indirection in the brick data layout permits distinct physical and logical data layouts; we can
therefore store the bricks in memory to reduce the data movement of packing and unpacking during cross-node
communication.  We have demonstrated strong scaling by reducing communication time.  

\paragraph{Next Steps}

We will continue to work with ECP application teams to integrate our tools with their efforts.  In particular,
we are integrating bricks into the Proto system, used in subsurface flows.
Moreover, we are developing online tuning capability for ytopt/SuRF. 
