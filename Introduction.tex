\section{Introduction}
The Exascale Computing Project Software Technology (ECP ST) focus area represents the key bridge between Exascale systems and the scientists developing applications that will run on those platforms. ECP offers a unique opportunity to build a coherent set of software (often referred to as the ``software stack'') that will allow application developers to maximize their ability to write highly parallel applications, targeting multiple Exascale architectures with runtime environments that will provide high performance and resilience. But applications are only useful if they can provide scientific insight, and the unprecedented data produced by these applications require a complete analysis workflow that includes new technology to scalably collect, reduce, organize, curate, and analyze the data into actionable decisions. This requires approaching scientific computing in a holistic manner, encompassing the entire user workflow—from conception of a problem, setting up the problem with validated inputs, performing high-fidelity simulations, to the application of uncertainty quantification to the final analysis. The software stack plan defined here aims to address all of these needs by extending current technologies to Exascale where possible, by performing the research required to conceive of new approaches necessary to address unique problems where current approaches will not suffice, and by deploying high-quality and robust software products on the platforms developed in the Exascale systems project.
The ECP ST portfolio has established a set of interdependent projects that will allow for the research, development, and delivery of a comprehensive software stack, as summarized in Table~\ref{table:wbs}.

\begin{table}
	\begin{tabular}{|>{\columncolor[gray]{0.8}}p{0.10\linewidth}|>{\columncolor[rgb]{0.88,1,1}}p{0.15\linewidth}|p{0.6\linewidth}|}\hline
	    \vfill WBS 2.3.1\vfill & \vfill \centering{Programming Models and Runtimes} \vfill & \vfill Cross-platform, production-ready programming infrastructure to support development and scaling of mission-critical software at both the node and full-system levels.\vfill \\\hline
		\vfill WBS 2.3.2 \vfill & \vfill \centering{Development Tools} \vfill & \vfill A suite of tools and supporting unified infrastructure aimed at improving developer productivity across the software stack. This scope includes debuggers, profilers, and the supporting compiler infrastructure, with a particular emphasis on LLVM~\cite{LLVM:2018} as a delivery and deployment vehicle. \vfill \\\hline
		\vfill WBS 2.3.3 \vfill & \vfill \centering{Mathematical Libraries} \vfill & \vfill Mathematical libraries and frameworks that (i) interoperate with the ECP software stack; (ii) are incorporated into ECP applications; and (iii) provide scalable, resilient numerical algorithms that facilitate efficient simulations on Exascale computers.\vfill \\\hline
		\vfill WBS 2.3.4 \vfill & \vfill \centering{Data and Visualization} \vfill & \vfill Production infrastructure necessary to manage, share, and facilitate analysis and visualization of data in support of mission-critical codes. Data analytics and visualization software that supports scientific discovery and understanding, despite changes in hardware architecture and the size, scale, and complexity of simulation and performance data produced by Exascale platforms. \vfill \\\hline
		\vfill WBS 2.3.5 \vfill & \vfill \centering{Software Ecosystem and Delivery} \vfill & \vfill Development and coordination of Software Development Kits (SDKs), the Extreme-scale Scientific Software Stack (E4S) across all of ECP ST projects.  Development of capabilities in Spack~\cite{gamblin+:sc15} in collaboration with NNSA's primary sponsorship.  Development of SuperContainers~\cite{Supercontainers} and coordination of container-based workflows across DOE computing faciities.\vfill \\\hline
		\vfill WBS 2.3.6 \vfill & \vfill \centering{NNSA ST} \vfill & \vfill Development and enhancement of open source software capabilities that are primarily developed at Lawrence Livermore, Los Alamos and Sandia National Laboratories.  Funds for engaging open science application and software teams in the use and enhancement of these products.\vfill \\\hline
	\end{tabular}
	\caption{\label{table:wbs} ECP ST Work Breakdown Structure (WBS), Technical Area, and description of scope.}
\end{table}

ECP ST is developing a software stack to meet the needs of a broad set of Exascale applications. The current software portfolio covers many projects spanning the areas of programming models and runtimes, development tools, mathematical libraries and frameworks, data management, analysis and visualization, and software delivery. The ECP software stack was developed bottom up based on application requirements and the existing software stack at DOE HPC Facilities. The portfolio comprises projects selected in two different ways: 
\begin{enumerate}
\item Thirty projects funded by the DOE Office of Science (ASCR) that were selected in October 2016 via an RFI and RFP process, considering prioritized requirements. The initial collection of loosely coupled projects has been re-organized twice and is now in a form that should serve us well as we more to the more formal execution phases of the project.
\item Three DOE NNSA/ASC funded projects that are part of the Advanced Technology Development and Mitigation (ATDM) program, which is in its sixth year (started in FY14). These projects are focused on longer term research to address the shift in computing technology to extreme, heterogeneous architectures and to advance the capabilities of NNSA/ASC simulation codes. 
\end{enumerate}
Since the initial selection process, ECP ST has reorganized efforts as described in Section~\ref{subsect:ProjectRestructuring}.

\begin{figure}
	\centering
	\includegraphics[width=0.9\linewidth]{ECP21}
	\caption{The ECP Work Breakdown Structure through Level 3 (L3). Under Software Technology, WBS 2.3.6 consolidates ATDM contributions to ECP into a new L3 area.}
	\label{fig:ecp2}
\end{figure}

\subsection{Background}
Historically, the software used on supercomputers has come from three sources: computer system vendors, DOE laboratories, and academia. Traditionally, vendors have supplied system software:  operating system, compilers, runtime, and system-management software. The basic system software is typically augmented by software developed by the DOE HPC facilities to fill gaps or to improve management of the systems. An observation is that it is common for system software to break or not perform well when there is a jump in the scale of the system.
 
Mathematical libraries and tools for supercomputers have traditionally been developed at DOE laboratories and universities and ported to the new computer architectures when they are deployed. These math libraries and tools have been remarkably robust and have supplied some of the most impactful improvements in application performance and productivity. The challenges have been the constant adapting and tuning to rapidly changing architectures.
 
Programming paradigms and the associated programming environments that include compilers, debuggers, message passing, and associated runtimes have traditionally been developed by vendors, DOE laboratories, and universities. The same can be said for file system and storage software. An observation is that the vendor is ultimately responsible for providing a programming environment and file system with the supercomputer, but there is often a struggle to get the vendors to support software developed by others or to invest in new ideas that have few or no users yet. Another observation is that file-system software plays a key role in overall system resilience, and the difficulty of making the file-system software resilient has grown non-linearly with the scale and complexity of the supercomputers.
 
In addition to the lessons learned from the traditional approaches, Exascale computers pose unique software challenges including the following.
\begin{itemize}
\item \textbf{Extreme parallelism:} Experience has shown that software breaks at each shift in scale. Exascale systems are predicted to have a billion-way concurrency almost exclusively from discrete accelerator devices, similar to today's GPUs. Because clock speeds have essentially stalled, the 1000-fold increase in potential performance going from Petascale to Exascale is entirely from concurrency improvements.
\item \textbf{Data movement in a deep memory hierarchy: }Data movement has been identified as a key impediment to performance and power consumption. Exascale system designs are increasing the types and layers of memory, which further challenges the software to increase data locality and reuse, while reducing data movement.
\item \textbf{Discrete memory and execution spaces:} The node architectures of Exascale systems include host CPUs and discrete device accelerators.  Programming for these systems requires coordinated transfer of data and work between the host and device. While some of this transfer can be managed implicitly, for the most performance-sensitive phases, the programmer typically must manage host-device coordination explicitly.  Much of the software transformation effort will be focused on this issue.
\end{itemize}
 
In addition to the software challenges imposed by the scale of Exascale computers, the following additional requirements push ECP away from the historical approaches for getting the needed software for DOE supercomputers.
\begin{itemize}
\item \textbf{2021 acceleration:} ECP has a goal of accelerating the development of the U.S. Exascale systems and enabling the first deployment by 2021. This means that the software needs to be ready sooner, and the approach of just waiting until it is ready will not work. A concerted plan that accelerates the development of the highest priority and most impactful software is needed.
\item \textbf{Productivity:} Traditional supercomputer software requires a great deal of expertise to use. ECP has a goal of making Exascale computing accessible to a wider science community than previous supercomputers have been. This requires the development of software that improves productivity and ease of use.
\item \textbf{Diversity:} There is a strong push to make software run across diverse Exascale systems. Accelerator devices from Nvidia have been available for many years and specific host-device programming and execution applications have been successfully ported to these platforms.  Exascale platforms will continue to have this kind of execution model, but with different programming and runtime software stacks.  Writing high-performance, portable code for these platforms will be challenging.
\item \textbf{Analytics and machine learning:} Future DOE supercomputers will need to solve emerging data science and machine learning problems in addition to the traditional modeling and simulation applications. This will require the development of scalable, parallel analytics and machine learning software for scientific applications, much of which does not exist today.
\end{itemize}
 
The next section describes the approach employed by ECP ST to address the Exascale challenges.

\subsection{ECP Software Technology Architecture and Design}
ECP is taking an approach of codesign across all its principal technical areas: applications development (AD), software technology (ST), and hardware \& integration (HI). For ECP ST, this means its requirements are based on input from other areas, and there is a tight integration of the software products both within the software stack as well as with applications and the evolving hardware. 

The portfolio of projects in ECP ST is intended to address the Exascale challenges and requirements described above. We note that ECP is not developing the entire software stack for an Exascale system. For example, we expect vendors to provide the core software that comes with the system (in many cases, by leveraging ECP and other open-source efforts). Examples of vendor-provided software include operating system, file system, compilers for C, C++, Fortran, etc. (increasingly derived from the LLVM community ecosystem to which ECP contributes), basic math libraries, system monitoring tools, scheduler, debuggers, vendor’s performance tools, MPI (based on ECP-funded projects), OpenMP (with features from ECP-funded project), and data-centric stack components. ECP develops other, mostly higher-level software that is needed by applications and is not vendor specific. ECP-funded software activities are concerned with extreme scalability, exposing additional parallelism, unique requirements of Exascale hardware, and performance-critical components. Other software that aids in developer productivity is needed and may come from third-party open-source efforts.

The ST portfolio includes both ASCR and NNSA ATDM funded efforts. The MOU established between DOE-SC and NNSA has formalized this effort.  Whenever possible, ASCR and ATDM efforts are treated uniformly in ECP ST planning and assessment activities.

ST is also planning to increase integration within the ST portfolio through increased use of software components and application composition vs. monolithic application design. An important transition that ECP can accelerate is the increased development and delivery of reusable scientific software components and libraries. While math and scientific libraries have long been a successful element of the scientific software community, their use can be expanded to include other algorithms and software capabilities, so that applications can be considered more an aggregate composition of reusable components than a monolithic code that uses libraries tangentially.

To accelerate this transition, we need a greater commitment on the part of software component developers to provide reliable and portable software that users can consider to be part of the software ecosystem in much the same way users depend on MPI and compilers. At the same time, we must expect application developers to participate as clients and users of reusable components, using capabilities from components, transitioning away from (or keeping as a backup option) their own custom capabilities.

\subsubsection{The Extreme-scale Scientific Software Stack (E4S)}\label{subsubsect:e4s}
On November 8, 2018, ECP ST released version 0.1 of the Extreme-scale Scientific Software Stack, E4S (\url{http://e4s.io}). It released version 0.2 of E4S in January 2019 and version 1.0 in November 2019. E4S contains a collection of the software products to which ECP ST contributes.  E4S is the primary conduit for providing easy access to ECP ST capabilities for ECP and the broader community.  E4S is also the ECP ST vehicle for regression and integration testing across DOE pre-Exascale and Exascale systems.

\begin{figure}
		\centering
		\fbox{\includegraphics[width=0.9\linewidth]{E4S-Build-Tree}}
	\caption{Using Spack~\cite{gamblin+:ecp18-spack-tutorial}, E4S builds a comprehensive software stack.  As ECP ST efforts proceed, we will use E4S for continuous integration testing, providing developers with rapid feedback on regression errors and providing user facilities with a stable software base as we prepare for Exascale platforms.  This diagram shows how E4S builds ECP products via an SDK target (the math libraries SDK called xSDK in this example).  The SDK target then builds all product that are part of the SDK (see Figure~\ref{fig:sdk-definition1} for SDK groupings), first defining and building external software products. Green-labeled products are part of the SDK. The blue-label indicates expected system tools, in this case a particular version of Python.  Black-labeled products are expected to be previously installed into the environment (a common requirement and easily satisified).  Using this approach, a user who is interested in only SUNDIALS (a particular math library) can be assured that the SUNDIALS build will be possible since it is a portion of what E4S builds and tests.}
	\label{fig:e4s-build-tree}
\end{figure}

E4S has the following key features:
\begin{itemize}
	\item \textbf{The E4S suite is a large and growing effort to build and test a comprehensive scientific software ecosystem:} E4S V0.1 contained 25 ECP products.  E4S V0.2, release in January 2019 contained 37 ECP products and numerous additional products needed for a complete software environment.  Eventually E4S will contain all open source products to which ECP contributes, and all related products needed for a holistic environment.
	\item \textbf{E4S is not an ECP-specific software suite:}  The products in E4S represent a holistic collection of capabilities that contain the ever-growing SDK collections sponsored by ECP and all additional underlying software required to use ECP ST capabilities.  Furthermore, we expect the E4S effort to live beyond the timespan of ECP, becoming a critical element of the scientific software ecosystem.
	\item \textbf{E4S is partitionable:} E4S products are built and tested together using a tree-based hierarchical build process.  Because we build and test the entire E4S tree, users can build any subtree of interest, without building the whole stack (see Figure~\ref{fig:e4s-build-tree}).
	\item \textbf{E4S uses Spack:} The Spack~\cite{gamblin+:ecp18-spack-tutorial} meta-build tool invokes the native build process of each product, enabling quick integration of new products, including non-ECP products.
	\item \textbf{E4S is available via containers:} In addition to a build-from-source capability using Spack, E4S maintains several container environments (Docker, Singularity, Shifter, CharlieCloud) that provides the lowest barrier to use.  Container distributions dramatically reduce installation costs and provide a ready-made environment for tutorials that leverage E4S capabilities.  For example, the ECP  application project CANDLE (Cancer Deep Learning Environment) uses an E4S container to provide a turnkey tutorial execution environment.
	\item \textbf{E4S distribution:} E4S products are available at \url{http://e4s.io}.
	\item \textbf{E4S developer community resources:} Developers interested in participating in E4S can visit the E4S-Project GitHub community at \url{https://github.com/E4S-Project}.	
\end{itemize}

\begin{figure}
	\centering
	\fbox{\includegraphics[width=0.9\linewidth]{E4S-Summary}}
	\caption{The Extreme-scale Scientific Software Stack (E4S) provides a complete Linux-based software stack that is suitable for many scientific workloads, tutorial and development environments.  At the same time, it is an open software architecture that can expand to include any additional and compatible Spack-enabled software capabilities. Since Spack packages are available for many products and easily created for others, E4S is practically expandable to include almost any robust Linux-based product.  Furthermore, E4S capabilities are available as subtrees of the full build: E4S is not monolithic.}
	\label{fig:e4s-is-isnot}
\end{figure}

\begin{figure}
	\centering
	\includegraphics[width=0.9\linewidth]{E4S-Build-Cache-Binaries}
	\caption{Using Spack build cache features, E4S builds can be accelerated by use of cached binaries for any build signature that Spack has already seen.}
	\label{fig:e4s-build-cache}
\end{figure}

\begin{figure}
	\centering
	\includegraphics[width=0.9\linewidth]{E4S-AWS-public-image}
	\caption{E4S is available as an Amazon AWS public image.  Images on Google and Microsoft Cloud environments will be available soon.}
	\label{fig:e4s-build-cache}
\end{figure}


The E4S effort is described in further detail in Sections~\ref{subsect:ecosystem}, especially Section~\ref{subsubsect:sdks}.

\subsubsection{Software Development Kits}\label{subsubsect:sdks}
One opportunity for a large software ecosystem project such as ECP ST is to foster increased collaboration, integration and interoperability among its funded efforts. Part of ECP ST design is the creation of software development kits (SDKs).  SDKs are collections of related software products (called packages) where coordination across package teams will improve usability and practices and foster community growth among teams that develop similar and complementary capabilities. SDKs have the following attributes:
\begin{table}
	\begin{mdframed}
\begin{enumerate}
	\item \textbf{Domain scope:} Each SDK will be composed of packages whose capabilities are within a natural functionality domain. Packages within an SDK provide similar capabilities that can enable leveraging of common requirements, design, testing and similar activities. Packages may have a tight complementary such that ready composability is valuable to the user.
	\item \textbf{Interaction models:} How packages within an SDK interact with each other. Interactions include common data infrastructure, or seamless integration of other data infrastructures; access to capabilities from one package for use in another.
	\item \textbf{Community policies:} Expectations for how package teams will conduct activities, the services they provide, software standards they follow, and other practices that can be commonly expected from a package in the SDK.
	\item \textbf{Meta-build system:} Robust tools and processes to build (from source), install and test the SDK with compatible versions of each package. This system sits on top of the existing build, install and test capabilities for each package.
	\item \textbf{Coordinated plans:} Development plans for each package will include efforts to improve SDK capabilities and lead to better integration and interoperability.
	\item \textbf{Community outreach:} Efforts to reach out to the user and client communities will include explicit focus on SDK as product suite.
\end{enumerate}
	\end{mdframed}
\caption{\label{table:sdk-attributes} Software Development Kits (SDKs) provide an aggregation of software products that have complementary or similar attributes.  ECP ST uses SDKs to better assure product interoperability and compatibility.  SDKs are also essential aggregation points for coordinated planning and testing. SDKs are an integral element of ECP ST~\cite{Heroux-SDK-Podcast}.  Section~\ref{subsubsect:ecosystem-sdk} describes the six SDK groupings and the current status of the SDK effort.}
\end{table}

\paragraph{ECP ST SDKs}
As part of the delivery of ECP ST capabilities, we will establish and grow a collection of SDKs. The new layer of aggregation that SDKs represent are important for improving all aspects of product development and delivery. The communities that will emerge from SDK efforts will lead to better collaboration and higher quality products. Established community policies will provide a means to grow SDKs beyond ECP to include any relevant external effort. The meta-build systems (based on Spack) will play an important role in managing the complexity of building the ECP ST software stack, by providing a new layer where versioning, consistency and build options management can be addressed at a mid-scope, below the global build of ECP ST products.
Each ECP ST L3 (five of them) has funds for an SDK project from which we have identified a total of six SDKs and an at-large collection of remaining products that will be delivered outside of the SDK grouping.  Section~\ref{subsubsect:ecosystem-sdk} provides an update on the progress in defining SDK groupings. For visibility, we provide the same diagram in Figure~\ref{fig:sdk-definition1-0}.

\begin{figure}[htb]
	\centering
	\fbox{\includegraphics[width=6.5in]{projects/2.3.5-Ecosystem/2.3.5.01-Ecosystem-SDK/SDKdefinition2}}
	\caption{\label{fig:sdk-definition1-0}The above graphic shows the breakdown of ECP ST products into 6 SDKs ( the first six columns).  The rightmost column lists products that are not part of an SDK, but are part of Ecosystem group that will also be delivered as part of E4S. The colors denoted in the key map all of the ST products to the ST technical area they are part of.  For example, the xSDK consists of products that are in the Math Libraries Technical area, plus TuckerMPI which is in the Ecosystem and Delivery technical area.  Section~\ref{subsubsect:ecosystem-sdk} provides an update on the progress in defining SDK groupings.}
\end{figure}


\subsubsection{ECP ST Product Dictionary}
In the past year, ECP has initiated an effort to explicitly manage ECP ST products and their dependencies (see Section~\ref{subsubsect:dep-management}).  In order to eliminate ambiguities, we first need a product dictionary: an official list of publicly-name products to which ECP ST teams contribute their capabilities and upon which ECP ST clients depend.  The ECP Product Dictionary is single, managed table.  It presently contains 70 primary products along with subproducts that are either components within a product or particular implementations if a standard API.  Two special primary products are the Facilities stack and Vendor stack.  Having these stacks on the list enables ST teams to indicate that their capabilities are being delivered to ecosystems outside of ECP.

Figure~\ref{fig:product-dictionary-overview} describes the policy for ECP ST teams to add and manage their contributions to the Product Dictionary.  Figure~\ref{fig:product-dictionary} shows a snapshot of the beginning and end of the current ECP ST Product Dictionary, which is maintained on the ECP Confluence wiki.

\begin{figure}
	\centering
	\fbox{\includegraphics[width=0.9\linewidth]{ProductDictionaryOverview}}
	\caption{This figure shows a screenshot from the top of the ECP Confluence wiki page containing the ECP ST Product Dictionary.  The Product Dictionary structure contains primary and secondary products.  Client (consumer) dependencies are stated against the primary product names only, enabling unambiguous mapping of AD-on-ST and ST-on-ST dependencies.}
	\label{fig:product-dictionary-overview}
\end{figure}

\begin{figure}
	\centering
	\fbox{\includegraphics[width=0.9\linewidth]{ConfluenceProductDictionaryExample}}
	\caption{These screen shots are from the ECP Confluence Product Dictionary Table.  The table is actively managed to include primary and secondary products to which ECP ST team contribute and upon which ECP ST clients depend.  Presently the Product Dictionary contains 70 primary products.  Secondary products are listed under the primary product with the primary product as a prefix.  For example, AID is the second listed primary product in this figure.  STAT, Archer and FLIT are component subproducts.  MPI (not shown) is another primary product.  MPICH and OpenMPI are two robust MPI implementations and are listed as MPI subproducts.}
	\label{fig:product-dictionary}
\end{figure}

\subsubsection{ECP Product Dependency Management}\label{subsubsect:dep-management}
Given the ECP ST Product Dictionary, and a similar dictionary for ECP AD and Co-Design products, ECP as a project has created a dependency database that enabled creation and characterization of product-to-product dependencies.  ECP manages these dependencies in a Jira database using a custom Jira issue type, Dependency.  The dependency database provides an important tool for understanding and managing ECP activities.  The dependency information is valuable both within and outside the project.  Figure

\begin{figure}
	\centering
	\fbox{\includegraphics[width=0.9\linewidth]{DependencyDashboard-EditPanel}}
	\caption{Using Jira, ECP manages its AD, ST, HI, vendor and facilities dependencies.  This figure shows a dashboard snapshot along with an edit panel that support creation and management of a consumer-on-producer dependency.}
	\label{fig:dependency-dashboard-edit}
\end{figure}

\begin{figure}
	\centering
	\includegraphics[width=0.9\linewidth]{PETSc-TAO-Dependencies}
	\caption{This query result from the ECP Jira Dependency database lists all consumers of capabilities from the PETSc/TAO product.  By selecting the details of one of the dependency issues, one can further see how critical the dependency is and see any custom information peculiar to the particular dependency.}
	\label{fig:petsc-tao-dependencies}
\end{figure}

\subsubsection{ECP ST Software Delivery}
An essential activity for, and the ultimate purpose of, ECP ST is the delivery of a software stack that enables productive and sustainable Exascale computing capabilities for target ECP applications and platforms, and the broader high-performance computing community. The ECP ST Software Ecosystem and Delivery sub-element (WBS 2.3.5) and the SDKs in each other sub-element provide the means by which ECP ST will deliver its capabilities.
\paragraph{ECP ST Delivery and HI Deployment}
Providing the ECP ST software stack to ECP applications requires coordination between ECP ST and ECP HI. The focus areas have a complementary arrangement where ECP ST delivers its products and ECP HI deploys them. Specifically:
\begin{itemize}
	\item ST \textbf{delivers} software.  ECP ST products are delivered directly to application teams, to vendors and to facilities.  ECP ST designs and implements products to run on DOE computing facilities platforms and make products available as source code via GitHub, GitLab or some other accessible repository.
	\item HI facilitates efforts to \textbf{deploy} ST (and other) software on Facilities platforms by installing it where users expect to find it. This could be in /usr/local/bin or similar directory, or available via “module load”.
\end{itemize}
Separating the concerns of delivery and deployment is essential because these activities require different skill sets. Furthermore, ECP ST delivers its capabilities to an audience that is beyond the scope of specific Facilities’ platforms. This broad scope is essential for the sustainability of ECP ST products, expanding the user and developer communities needed for vitality. In addition, ECP HI, the computer system vendors and other parties provide deployable software outside the scope of ECP ST, therefore having the critical mass of skills to deploy the entire software stack.

\paragraph{ECP ST Delivery Strategy}
ECP ST delivers it software products as source code, primarily in repositories found on GitHub, Gitlab installations or similar platforms. Clients such as ECP HI, OpenHPC and application developers with direct repository access then take the source and build, install and test our software. The delivery strategy is outlined in Figure~\ref{fig:softwarestack}.  

Users access ECP ST products using these basic mechanisms (see Figure~\ref{fig:productsoverview} for deliverable statistics):
\begin{itemize}
	\item \textbf{Build from source code:} The vast majority of ECP ST products reach at least some of their user base via direct source code download from the product repository.  In some cases, the user will download a single compressed file containing product source, then expand the file to expose the collection of source and build files.  Increasingly, users will fork a new copy of an online repository.  After obtaining the source, the user executes a configuration process that detects local compilers and libraries and then builds the product.  This kind of access can represent a barrier for some users, since the user needs to build the product and can encounter a variety of challenges in that process, such as an incompatible compiler or a missing third-party library that must first be installed.  However, building from source can be a preferred approach for users who want control over compiler settings, or want to adapt how the product is used, for example, turning on or off optional features, or creating adaptations that extend product capabilities.  For example, large library frameworks such as PETSc and Trilinos have many tunable features that can benefit from the user building from source code.  Furthermore, these frameworks support user-defined functional extensions that are easier to support when the user builds the product from source.  ECP ST is leveraging and contributing to the development of Spack~\cite{gamblin+:sc15}.  Via meta-data stored in a Spack \textit{package} defined for each product, Spack leverages a product's native build environment, along with knowledge about its dependencies, to build the product and dependencies from source.  Spack plays a central role in ECP ST software development and delivery processes by supporting turnkey builds of the ECP ST software stack for the purposes of continuous integration testing, installation and seamless multi-product builds.
	\item \textbf{DOE computing facilities:} Each DOE computing facility (ALCF, OLCF, NERSC, LLNL and ACES [LANL/SNL]) provides pre-built versions of 17 to 20 ECP ST products (although the exact mix of products varies somewhat at each site).  Many of these products are what users would consider to be part of the core system capabilities, including compilers, e.g., Flang (Section~\ref{subsubsect:flang}) and LLVM (Section~\ref{subsubsect:sollve}), and parallel programming environments such as MPICH (Section~\ref{subsubsect:mpich}), OpenMPI (Section~\ref{subsubsect:openmpi}) and OpenMP (Section~\ref{subsubsect:bolt}).  Development tools such as PAPI (Section~\ref{subsubsect:exapapi}) and TAU (Section~\ref{subsubsect:tau}) are often part of this suite, if not already included in the vendor stack. Math and data libraries such as PETSc (Section~\ref{subsubsect:petsc}), Trilinos (Section~\ref{subsubsect:peeks}), HDF5 (Section~\ref{subsubsect:exahdf5}) and others are also available in some facilities software installations.  We anticipate and hope for increased collaboration with facilities via the ECP Hardware \& Integration (HI) Focus Area.  We are also encouraged by multi-lab efforts such as the Tri-Lab Operating System Stack (TOSS)~\cite{TOSS} that are focused on improving uniformity of software stacks across facilities.
	\item \textbf{Vendor stacks:} Computer system vendors leverage DOE investments in compilers, tools and libraries.  Of particular note are the wide use of MPICH(Section~\ref{subsubsect:mpich}) as software base for most HPC vendor MPI implementations and the requirements, analysis, design and prototyping that ECP ST teams provide.  Section~\ref{subsection:external-contributions} describes some of these efforts.
	\item \textbf{Binary distributions:} Approximately 10 ECP ST products are available via binary distributions such as common Linux distributions, in particular via OpenHPC\cite{OpenHPC}.  ECP ST intends to foster growth of availability via binary distributions as an important way to increase the size of the user community and improve product sustainability via this broader user base.
\end{itemize}

\begin{figure}
	\centering
	\includegraphics[width=0.9\linewidth]{SoftwareStack}
	\caption{\textbf{The ECP ST software stack is delivered to the user community through several channels.} Key channels are via source code, increasingly using SDKs, direct to Facilities in collaboration with ECP HI, via binary distributions, in particular the OpenHPC project and via HPC vendors.  The SDK leadership team includes  ECP ST team members with decades of experience delivering scientific software products.}
	\label{fig:softwarestack}
\end{figure}

\subsection{ECP ST Project Restructuring}\label{subsect:ProjectRestructuring}

The initial organization of ECP ST was based on discussions that occurred over several years of Exascale planning within DOE, especially the DOE Office of Advanced Scientific Computing Research (ASCR).  Figure~\ref{fig:ecpstv1} shows the conceptual diagram of this first phase.  The 66 ECP ST projects were mapped into 8 technical areas, in some cases arbitrating where a project should go based on its primary type of work, even if other work was present in the project.  In November 2017, ECP ST was reorganized into 5 technical areas, primarily through merging a few smaller areas, and the number of projects was reduced to 56 (presently 55 due to further merging in \ecosystem).  Figure~\ref{fig:ecpstv2} shows the diagram of the second phase of ECP ST.  With the CAR V2.0, we will describe the next phase of organization refinement needed to best position ECP ST for success in the CD-2 phase of the project.

\begin{figure}
	\centering
	\includegraphics[width=0.9\linewidth]{STFY20WBS}
	\caption{\label{fig:wbs-FY20} The FY20 ECP ST WBS structure includes a new L3 (2.3.6) and better balances L4 subprojects in the first four L3 technical areas.  Technical area 2.3.5 has only two projects, which are focused on meta-product development in SDKs, E4S, Spack and SuperContainers.}
\end{figure}

\begin{figure}
\begin{mdframed}
\begin{itemize}
\item Phase 1: 66 total L4 subprojects
\begin{itemize}
\item 35 projects funded by the DOE Office of Science that were selected in late 2016 via an RFI and RFP process, considering prioritized requirements of applications and DOE facilities. 
These projects started work in January–March 2017 depending on when the contracts were awarded.
\item 31 ongoing DOE NNSA funded projects that are part of the Advanced Technology Development and Mitigation (ATDM) program. The ATDM program started in FY14.  These projects are focused on longer term research to address the shift in computing technology to extreme, heterogeneous architectures and to advance the capabilities of NNSA simulation codes.
\end{itemize}
\item Phase 2: 55 total L4 subprojects
\begin{itemize}
\item 41 ASCR-funded projects.  Added  2 \ecosystem\ projects and 4 SDK projects.
\item 15 ATDM projects: Combined the previous 31 ATDM projects into one project per technical area per lab.  ATDM projects are generally more vertically integrated and would not perfectly mapped to any proposed ECP ST technical structure.  Minimizing the number of ATDM projects within the ECP WBS structure reduces complexity of ATDM to ECP coordination and gives ATDM flexibility in revising its portfolio without disruption to the ECP-ATDM mapping.
\end{itemize}
\item Phase 3: 33 total L4 subprojects.  Fewer, larger and more uniform-sized projects
\begin{itemize}
	\item Starting with FY2020, ECP ST has further consolidated L4 projects to foster additional synergies and amortize project overheads as ECP heads into Critical Decision Phase 2~\cite{413.3B}, where more rigor in planning and execution are needed.
	\item 5 L3s to 6: New NNSA ST L3
	\item 40 ST SC-funded L4 subprojects to 30.
	\begin{itemize}
	\item \pmr – 13 to 9, \tools - 6 to 6, \mathlibs - 7 to 6, \dataviz - 10 to 7, \ecosystem - 4 to 3.
	\item Includes 2 new L4 subprojects in SW Ecosystem.
	\end{itemize}
	\item 15 ST NNSA-funded projects transferred to new NNSA ST L3. Consolidated from 15 to 3 L4 subprojects.
	\item No more small subprojects.
	\item Figure~\ref{fig:wbs-FY20} show the overall structure.
\end{itemize}
\end{itemize}
\end{mdframed}

\caption{\label{fig:project-remapping}Project remapping summary from Phase 1 (through November 2017) to Phase 2 (November 2017 -- September 30, 2019) to Phase 3 (After October 1, 2019)}
\end{figure}


\begin{figure}
	\centering
	\includegraphics[width=0.9\linewidth]{ECPSTV1}
	\caption{ECP ST before November 2017 reorganization.  This conceptually layout emerged from several years of Exascale planning, conducted primarily within the DOE Office of Advanced Scientific Computing Research (ASCR).  After a significant restructuring of ECP that removed much of the facilities activities and reduced the project timeline from 10 to seven years, and a growing awareness of what risks had diminished, this diagram no longer represented ECP ST efforts accurately.}
	\label{fig:ecpstv1}
\end{figure}
\begin{figure}
	\centering
	\includegraphics[width=0.9\linewidth]{ECPSTV2}
	\caption{ECP ST after November 2017 reorganization.  This diagram more accurately reflects the priorities and efforts of ECP ST given the new ECP project scope and the demands that we foresee.}
	\label{fig:ecpstv2}
\end{figure}
\begin{figure}
	\centering
	\includegraphics[width=0.9\linewidth]{ECPSTV3}
	\caption{ECP ST after October 2019 reorganization.  This diagram reflects the further consolidation of NNSA open source contributions to enable more flexible management of NNSA ST contributions.}
\end{figure}
\begin{figure}
	\centering
	\includegraphics[width=0.9\linewidth]{ECP-ST-Leads}
	\caption{ECP ST Leadership Team as of October 2019.}
	\label{fig:ecpstleads}
\end{figure}

%\subsection{New Project Efforts}
%ECP ST is preparing several strategic changes for the FY2020 restructuring (Phase 3).   These changes will be reported in the CAR V2.0.  No other significant changes have been made since the CAR V1.0.


%\subsubsection{FFTs}\label{subsubsect:ffts}
%ECP ST has initiated two new efforts in Fast Fourier Transforms (FFTs).  FFTs provide an essential mathematical tool to many application areas.  From the very beginning of HPC, vendors have provided optimized FFT libraries for their users.  The advent of FFTW~\cite{FFTW05}, a tunable high-performance library with a well-designed interface, enabled a \textit{de facto} standardization of FFT interfaces, and an effective source base for vendor libraries, which adapted FFTW source for their platforms.
%
%There is some concern in the community that FFTW is no longer actively developed, nor well prepared for emerging platforms.  Furthermore, FFTW's strong copylefting license (which forces its users to make their own software open source in the same way) has always been a challenge to users.  While vendors are still committed to providing optimized FFT libraries, whether or not FFTW is available, we believe it is prudent to explore a new software stack and have funded a short-term project to explore this possibility.  The new library will also explore problem formulations that could significantly reduce the computational cost of FFTs.  This new effort, called FFTX, will be led by Lawrence Berkeley National Lab, under the existing math libraries project (Section~\ref{subsubsect:strumpack}).
%
%A second FFT project will address a consensus opportunity to provide a sustainable 3D FFT library built from established but \textit{ad hoc} software tools that have traditionally been part of application codes, but not extracted as independent, supported libraries.  These 3D FFTs rely on third-party 1D FFTs, either from FFTW or from vendor libraries.
%
%The goal of this second project, FFT-ECP, led by the University of Tennessee and integrated into one of its existing projects (Section~\ref{subsubsect:slate}) is to:
%\begin{itemize}
%\item Collect existing FFT capabilities recently made available from ECP application teams (LAMMPS/fftMPI and HACC/SWFFT). 
%\item Assess gaps and make available as a sustainable math library.
%\item Explore opportunities to build 3D FFT libraries on vendor 1D and 2D kernels, especially leveraging on-node concurrency from 2D and batched 1D formulations.
%\item Focus on capabilities for Exascale platforms.
%\item Emphasize leverage of vendor capabilities and addressing vendor deficiencies over creation of new and independent software stack.
%\end{itemize}
%
%This effort, while not addressing the concerns about FFTW directly, is essential to providing a new and sustainable FFT software stack that leverages the large investment by the broader HPC community in FFT software.  The payoff from this effort is almost guaranteed.  Also, should the FFTX project also go forward, it will provide an FFTW-compatible interface that would allow FFT-ECP to use FFTX as one option, in addition to external FFT libraries.
%
%\subsubsection{LLNL Math Libraries}\label{subsubsect:llnl-math-libs}
%When ECP ST started, some important capabilities were not part of the original portfolio, even though their engagement is essential for ECP application success.  This is true of the LLNL math library \textit{hypre}~\cite{hypre}.  This library is widely used to provide scalable multigrid preconditioners across several ECP applications.  Funding to support adaptation of \textit{hypre}  in preparation for Exascale platforms at science lab facilities, e.g., Argonne, and for ECP science applications was not part of the original ECP ST portfolio.  We have added funding for this effort, starting in June 2018.  In addition, we provided new funding for another LLNL math library, MFEM~\cite{mfem:homepage}, so that the MFEM team can participate in SDK efforts for math libraries.
