\subsubsection{\stid{6.01} LANL ATDM Software Ecosystem \& Delivery Projects - BEE/Charliecloud Subproject} 

\paragraph{Overview}
The BEE/Charliecloud subproject is creating software tools to increase portability
and reproducibility of scientific applications on high performance and cloud
computing platforms.  Charliecloud \cite{priedhorskyrrandlestc2016} is an unprivileged Linux container
runtime.  It allows developers to use the industry-standard Docker
\cite{dockerinc}
toolchain to containerize scientific applications and then execute them on
unmodified DOE facility computing resources without paying any performance
penalty.  BEE \cite{beeproject} (Build and Execution Environment) is a toolkit providing
users with the ability to execute application workflows across a diverse set of
hardware and runtime environments.  Using Bee's tools, users can build and
launch applications on HPC clusters and public and private clouds, in
containers or in containers inside of virtual machines, using a variety of
container runtimes such as Charliecloud and Docker. 

\paragraph{Key Challenges}
Other HPC-focused container runtimes exist, such as NERSC's Shifter
\cite{canonrsjacobsend} and
Singularity \cite{kurtzergmsochatvbauermw}.  These alternative runtimes have characteristics, such as
complex setup requirements and privileged user actions, that are undesirable in
many environments.  Nevertheless, they represent a sizable fraction of the
existing HPC container runtime mindshare.  A key challenge for BEE is
maintaining support for multiple runtimes and the various options that they require
for execution.  This is especially true in the case of Singularity, which
evolves rapidly.  Similarly, there is a diverse collection of resources that
BEE and Charliecloud must support to serve the ECP audience.  From multiple HPC
hardware architectures and HPC accelerators such as GPUs and FPGAs, to
differing HPC runtime environments and resource managers, to a multitude of
public and private cloud providers, there is a large set of available resources
that BEE and Charliecloud must take into consideration to provide a
comprehensive solution.

\paragraph{Solution Strategy}
The BEE/Charliecloud project is focusing first on providing support for
containerized production LANL scientific applications across all of the
existing LANL production HPC systems.  The BEE/Charliecloud components required
for production use at LANL will be documented, released and fully supported.
Follow-on development will focus on expanding support to additional DOE
platforms.  This will mean supporting multiple hardware architectures,
operating systems, resource managers, and storage subsystems.  Support for
alternative container runtimes, such as Docker, Shifter, and Singularity is
planned.

\paragraph{Recent Progress}
% previous recent progress
%Recent Charliecloud progress has been focusing on understanding and documenting
%best practices for running large scale MPI jobs using containerized runtimes.
%Additional work has been done to enhance support for using containers with
%GPUs.  Charliecloud is available at https://github.com/hpc/charliecloud and was
%recently approved for inclusion in the next release of OpenHPC.
%
%BEE currently has beta-level support for launching Charliecloud containers on
%LANL HPC systems.  Automated BEE scalability testing is nearing production
%readiness.
Recent Charliecloud progress has focused on understanding and documenting best
practices for running large scale MPI jobs using containerized runtimes.
Charliecloud is enhancing support for multiple MPI implementations.
Charliecloud is available at https://github.com/hpc/charliecloud and is
distributed inside of Debian and Gentoo Linux distributions as well as being
part of OpenHPC.  Charliecloud won an 2018 R\&D-100 award.

BEE fully supports launching Charliecloud containers on all LANL HPC systems.
It can also launch containers on AWS and OpenStack clouds such as NSF
Chameleon.  BEE also supports interactive launching of jobs with the SLURM
resource manager. BEE was shown at the end of FY19 to support a complex multiphysics application with setup, in situp visualization and checkpoint-restart on a production system at LANL.

\paragraph{Next Steps}
A refactoring of BEE to support an open standard is underway. Support for the Open Workflow standard will allow a base on a well defined workflow description language leveraged by other scientiic communities. This will then be tested on multiple systems to ensure portability.
