\subsubsection{\stid{2.04} SNL ATDM Tools} 

\paragraph{Overview}

The SNL ATDM Tools project is broken into two subprojects: the SNL ATDM DevOps subproject, and the Sandia ATDM Performance Analysis subproject.

The SNL ATDM DevOps subproject focus is on tools and processes supporting DevOps (Development Operations) for the ATDM software development efforts.
DevOps in the SNL ATDM context is all of the software infrastructure development, testing support, integration, and deployment work in support of the ATDM software application and component development teams.
The primary activities of this subproject are to (1) coordinate and prioritize tasks for the various teams that provide DevOps support for ATDM codes, applications, and customers, (2) develop and help deploy shared build, test, and install infrastructure across the ATDM codes and projects, (3) define and support development, testing, integration, and other related workflows for ATDM projects.

The SNL ATDM Performance Analysis subproject is scoped with providing a broad cross-section of performance-related support activities for the laboratories ATDM efforts.
These activities include: (1) providing support for high-performance, hardware-optimized cross-platform builds, including the generation of correct hardware compiler options/software defines; (2) performance
analysis of benchmarking runs, including thread and node scaling, on relevant ASC testbeds and platforms, and (3) provision for algorithm/code modification or editing of run scripts to optimize performance where issues are identified.

The SNL ATDM Performance Analysis subproject also develops profiling and correctness tools which work with the Kokkos Profiling hooks API.
These tools have been developed to provide insight into the timing of kernels written using Kokkos, as well as data structures utilizing Kokkos parallel containers or Views.
In a number of cases, the profiling tools act as {\em connectors}, establishing a link between important Kokkos performance events and vendor provided tools such as Intel's VTune, NVIDIA's NSight and Arm's MAP profilers.

\paragraph{Key Challenges}

The key challenges associated with this project are the extremely aggressive porting and optimization requirements associated with Sandia's ATDM efforts.
These activities are attempting to port and help support a minimum of three production applications, as well as multiple mini-applications and research prototypes to several ASC-relevant platforms.
The first-of-a-kind algorithms being used on these platforms produce complex interactions in the applications that must be fully studied and analyzed to ensure a high level of performance is being offered to the Sandia's user base.

Combined with the application development effort, Sandia is investing heavily in the development of the Trilinos scalable solver stack (used by several codes in ECP and the broader HPC community).
The Performance Analysis activity within ATDM is also providing low-level kernel and runtime optimization insight to developers in the Kokkos and Trilinos projects.
The DevOps activity within SNL ATDM is providing configuration, build, testing, and workflows tools and processes to keep this stack of software working on the variety of platforms and configurations.

\paragraph{Solution Strategy}

The SNL ATDM Tools has the following primary thrusts:

\begin{enumerate}

\item \textbf{Common Build, Test, and Integration Tools} ensure scalable DevOps efforts and support.

\item \textbf{Testing and Integration Workflows} ensure smooth and productive development and deployment efforts for ATDM software on target platforms.

\item \textbf{High-Performance Applications} ensure well optimized application, library and kernel performance across ASC-relevant computing architectures.

\item \textbf{Performance Portability} ensures performance portability of Sandia ATDM codes across diverse ASC-relevant computing architectures.

\item \textbf{Lightweight Performance Tool Infrastructure} ensures that lightweight tools exist for rapid performance analysis or performance issue identification.

\end{enumerate}

\noindent
The Sandia Performance Analysis sub project was formed from the older Performance Modeling and Analysis Team in 2015.
It's scope was refined to focus specifically on supporting application development activities at the laboratories, with the intent to help provide much stronger levels of performance across the Sandia software portfolio.
The project has provided significant application support since 2015 on topics including application porting and scaling on the ASC Trinity platform, porting to the ASC CTS-1 commodity clusters and has most recently been providing support for the forthcoming ASC Sierra platform housed at LLNL.

The Kokkos Profiling tools collection was formed in 2015 resulting from research efforts in several successful LDRD projects.
The experimental interface to Kokkos was prototyped in 2014/5 and has since been the default configuration when compiling the Kokkos library.

\paragraph{Recent Progress}

For FY17, the SNL ATDM DevOps subproject completed a number of results: (1) created a TriBITS prototype build and test system for SPARC; (2) set up ATDM project and issue tracking utilizing JIRA and JIRA Portfolio; (3) worked to stabilize Trilinos for ATDM customers, and (4) worked with contractor Kitware to improve CMake/CTest/CDash and adding Fortran support in Ninja, performance, enhanced CDash.
The Sandia Performance Analysis subproject provided the Sandia SPARC and EMPIRE applications with performance results on: (1) Intel Knights Landing many-core processors; (2) Intel Haswell multi-core processors; (3) IBM POWER8/NVIDIA P100 CORAL development systems; (4) ASC CTS-1 Broadwell processors, and, (5) early access ARM processors.
The results of benchmarking and cross-platform kernel benchmark times were reported to developers with ATDM including the project leads for SPARC and EMPIRE as well as the Trilinos solver project.

For the first half of FY18, the DevOps subproject (1) developed integration workflows with Trilinos for the ATDM SPARC and EMPIRE Apps to shield them from instability in Trilinos and yet still drive co-development with Trilinos; and (2) set up initial ATDM builds of Trilinos for EMPIRE configuration submitting to CDash and addressed native Trilinos test suite failures.
The Performance Analysis team has provided benchmark kernel timings for some important classes of kernels on: (1) Intel Skylake multi-core processors; (2) early-access IBM POWER9/NVIDIA Volta platforms (for CORAL activities), and, (3) additional ARM processors.
The compilation flags and environment configurations have been integrated into Kokkos and Trilinos for wider community use.

\paragraph{Next Steps}

Our next efforts are:

\begin{enumerate}

\item \textbf{Complete upgrade of CMake/CTest/CDash}: Upgraded CMake/CTest/CDash will provide for faster builds and tests, better display and better query capability on CDash.

\item \textbf{Complete ATDM Trilinos builds}: Setup of testing on rest of platforms used by EMPIRE and then extend the ATDM Trilinos build configuration for SPARC.

\item \textbf{Transition ATDM APPs to use ATDM Trilinos builds:} SPARC and EMPIRE builds will use the same standard ATDM Trilinos build reducing duplicate work maintaining these builds, less computing resources to run the build, etc.

\item \textbf{Detailed Build Timing:} during FY18 the Performance Analysis subproject is investigating the timing of complex application builds including the time spent in file input/output, compilation itself, directory traversal and other metrics.
The intention is to identify areas of optimization that will improve developer productivity (by reducing wait for builds to complete). 

\item \textbf{Additional Benchmarking:} additional platforms and more extensive benchmarking activities are currently underway, particularly on CORAL POWER9 development systems.
These studies will have improve the ``day-one'' performance of Sandia's application portfolio on the pre-Exascale Sierra platform when it is released to users during 2018.

\end{enumerate}
