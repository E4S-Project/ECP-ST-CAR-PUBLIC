\subsection{\stid{2} \tools}\label{subsect:tools}

\textbf{End State:}	A suite of compilers and development tools aimed at improving developer productivity across increasingly complex heterogeneous architectures, primarily focused on those architectures expected for the upcoming Exascale platforms of Frontier and Aurora.

\subsubsection{Scope and Requirements}

For Exascale systems, the compilers, profilers, debuggers, and other software development tools must be increasingly sophisticated to give software developers insight into the behavior of not only the application and the underlying hardware but also the details corresponding to the underlying programming model implementation and supporting runtimes (e.g., capturing details of locality and affinity). These capabilities should be enhanced with further integration into the supporting compiler infrastructure and lower layers of the system software stack (e.g., threading, runtime systems, and data transport libraries), and hardware support. Most of the infrastructure will be released as open source, as many of them already are, with a supplementary goal of transferring the technology into commercial products (including reuse by vendors of ECP enhancements to LLVM, such as Fortran/Flang, or direct distributions by vendors of software on platforms). Given the diversity of Exascale systems architectures, some subset of the tools may be specific to one or more architectural features and is potentially best implemented and supported by the vendor; however, the vendor will be encouraged to use open APIs to provide portability, additional innovation, and integration into the tool suite and the overall software stack.

\subsubsection{Assumptions and Feasibility }

The overarching goal of improving developer productivity for Exascale platforms introduces new issues of scale that will require more lightweight methods, hierarchical approaches, and improved techniques to guide the developer in understanding the characteristics of their applications and to discover sources of the errors and performance issues. Additional efforts for both static and dynamic analysis tools to help identify lurking bugs in a program, such as race conditions, are also likely needed. The suite of needed capabilities spans interfaces to hardware-centric resources (e.g., hardware counters, interconnects, and memory hierarchies) to a scalable infrastructure that can collect, organize, and distill data to help identify performance bottlenecks and transform them into an actionable set of steps and information for the software developer. Therefore, these tools share significant challenges due to the increase in data and the resulting issues with management, storage, selection, analysis, and interactive data exploration. This increased data volume stems from multiple sources, including increased concurrency, processor counts, additional hardware sensors and counters on the systems, and increasing complexity in application codes and workflows.

Compilers obviously play a fundamental role in the overall programming environment but can also serve as a powerful entry point for the overall tool infrastructure. In addition to optimizations and performance profiling, compiler-based tools can help with aspects of correctness, establishing connections between programming model implementations and the underlying runtime infrastructures, and auto-tuning. In many cases, today's compiler infrastructure is proprietary and closed source, limiting the amount of flexibility for integration and exploration into the Exascale development environment. In addition to vendor compiler options, this project aims to provide an open source compiler capability (via the LLVM ecosystem) that can play a role in better supporting and addressing the challenges of programming at Exascale. 


\subsubsection{Objectives}

This project will design, develop, and deploy an Exascale suite of development tools for development, analysis, and optimization of applications, libraries, and infrastructure from the programming environments of the project.  The project will seek to leverage techniques for common and identified problem patterns and create new techniques for data exploration related to profiling and debugging and support advanced techniques such as autotuning and compiler integration. We will seek to establish an open-source compiler activity leveraging activities around the LLVM infrastructure. For tools, the overarching goal is to leverage and integrate the data measurement, acquisition, storage, and analysis and visualization techniques being developed in other projects of the software stack.These efforts will require collaboration and integration with system monitoring and various layers within the software stack.


\subsubsection{Plan}
Multiple projects will be supported under the tools effort. To ensure relevance to DOE missions, most of these efforts shall be DOE laboratory led and leverage and collaborate with existing activities within the broader HPC community. Initial efforts will focus on identifying the core capabilities needed by the selected ECP applications, components of the software stack, expected hardware features, and the selected industry activities from within the Hardware and Integration focus area. The supported projects will target and implement early versions of their software on both CORAL and APEX systems, with an ultimate target of production-ready deployment on the Exascale systems. Throughout this effort the applications teams and other elements of the software stack will evaluate and provide feedback on their functionality, performance, and robustness. These goals will be evaluated yearly (or more often as needed) based on milestones as well as joint milestone activities shared across the associated software stack activities by AD and HI focus areas.

\subsubsection{Risk and Mitigation Strategies}

A primary risk exists in terms of adoption of the various tools by the broader community, including support by system vendors. Past experience has shown that a combination of laboratory-supported open source software and vendor-optimized solutions built around standard APIs that encourage innovation across multiple platforms is a viable approach, and this will be undertaken. We will track this risk primarily via the risk register.

Given its wide use within a range of different communities, and its modular design principles, the project's open source compiler activities will focus on the use of the LLVM compiler ecosystem as a path to reduce both scope and complexity risks and leverage with an already established path for NRE investments across multiple vendors. The compilers and their effectiveness are tracked in the risk register. 
%
In fact, in the past year, ECP has created a fork of the \texttt{llvm-project} upstream repository (see \url{https://github.com/llvm-doe-org}) to capture, integrate, and test LLVM projects, and to serve as a risk mitigation option if other compilers are not working successfully on the target platforms.

Another major risk for projects in this area is the lack of low-level access to hardware and software necessary for using emerging architectural features. Many of these nascent architectural features have immature implementations and software interfaces that must be refined prior to release to the broader community. This project should be at the forefront of this interaction with early delivery systems. This risk is also tracked in the risk register for compilers, which are particularly vulnerable.

\subsubsection{Future Trends}

Future architectures are becoming more heterogeneous and complex~\cite{vetter:2018:extreme}. As such, the role of languages, compilers, runtime systems, and performance and debugging tools will becoming increasingly important for productivity and performance portability. 
%
In particular, our ECP strategy focuses on improving the open source LLVM compiler and runtime ecosystem; LLVM has gained considerable traction in the vendor software community, and it is the core of many existing heterogeneous compiler systems from NVIDIA, AMD, Intel, ARM, IBM, and others.  We foresee that this trend will continue, which is why we have organized the \tools\ technical area around LLVM-oriented projects.  
%
We expect for many of our contributions to LLVM to address these trends for the entire community and will persist long after ECP ends. 
%
For example, our contributions for directive-based features for heterogeneous computing (e.g., OpenMP, OpenACC) will not only provide direct capabilities to ECP applications, but it will also impact the redesign and optimization of the LLVM infrastructure to support heterogeneous computing.
%
In a second example, Flang (open source Fortran compiler for LLVM; [the second version is also known as F18]) will become increasingly important to the worldwide Fortran application base, as vendors find it easier to maintain and deploy to their own Fortran frontend (based on Flang).  
%
Furthermore, as Flang become increasingly robust, researchers and vendors developing new architectures will have immediate access to Flang, making initial Fortran support straightforward in ways similar to what we are seeing in Clang as the community C/C++ frontend.

%
