\subsubsection{\stid{2.10} PROTEAS-TUNE - LLVM-DOE: Creating and Maintaining a DOE Fork of LLVM}\label{s:llvm-doe}

\paragraph{Overview}

The ECP funds multiple projects that develop compiler technologies, based on the
popular, open-source LLVM compiler infrastructure project. This ecosystem allows
customization to meet the unique needs of ECP, and a level of well-established
mechanisms to deploy technologies through vendors and at DOE’s leadership
facilities. Importantly, this provides an alternative open-source compiler
ecosystem to those provided by the vendor, thus reducing the dependence on the
vendor’s compilers, timelines, and staff (Risk 10032 that ST product will not
function or meet performance targets).

In addition, most today’s vendors already rely on LLVM as the foundation for
their compiler ecosystems. This means ECP technology has a path back to vendors
via LLVM itself or through a DOE-/ECP-focused fork of LLVM’s open source
repository. This work will focus on deployment to reduce Risk 10020.

More broadly, there are eight LLVM-related projects supported by ECP that have
a risk of not being used if developers cannot easily access their contributions.
This fork of LLVM will provide an opportunity for these projects to work
collectively on establishing synergies, interoperability, address the unique
needs of ECP, and mechanisms for making contributions back into the mainstream
LLVM code base. The tasks to setup the DOE Fork of LLVM are:

\begin{enumerate}

\item Set up a fork of the llvm-project upstream repository (see \url{https://github.com/llvm-doe-org}).

\item Enable continuous integration for the fork on various hardware of       interests.

\item Enable LLVM ECP related projects to be able to push and test branches.

\item Setup status information for the continuous information results.

\end{enumerate}


\paragraph{Solution Strategy}

\begin{enumerate}

\item The DOE LLVM repository is setup on GitHub as a fork of the llvm-project
      main repository also hosted on GitHub. This makes it easier to have a
      seamless synchronization with the main repository and keep all the
      GitHub main-fork integrated features.

\item The GitHub repository is autmatically mirrored in the GitLab premium
      instance hosted at ORNL.

\item The continous integration takes advantage of the GitLab CI infrastructure.
      This infrastructure is available on several machines form the ExCL lab as
      well as on Summit and Theta.

\end{enumerate}


\paragraph{Recent Progress}

\begin{enumerate}
\item Fork is setup with an automatic mirroring with the upstream repository.
      The mirroring is using GitHub Actions.

\item A GitLab premimum instance is running at ORNL and mirror autmatically the
      GitHub repository. The base continuous integration is running nightly
      for the main branch of the repository on ExCL machines (Kold and Leconte).
\end{enumerate}


\paragraph{Next Steps}

\begin{enumerate}
\item Add continuous integration on more hardware (AMD explorer node in ExCL,
      Summit and Theta)
\item Enhance the continous integration with additional LLVM sub-projects.
\item Add test-suite to the CI (e.g. SOLLVE validation test-suite).
\end{enumerate}
